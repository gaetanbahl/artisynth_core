\documentclass{article}

% Add search paths for input files
\makeatletter
\def\input@path{{../}{../../}{../texinputs/}}
\makeatother

\usepackage{amsmath}
\usepackage{framed}

%\usepackage{layout}
%\usepackage{showframe}
%%
%% Default settings for artisynth
%%
\NeedsTeXFormat{LaTeX2e}
%%\ProvidesPackage{artisynthDoc}[2012/04/05]

\usepackage[T1]{fontenc}
\usepackage[latin1]{inputenc}
\usepackage{listings}
\usepackage{makeidx}
\usepackage{latexml}
\usepackage{graphicx}
\usepackage{framed}
\usepackage{color}

\newcommand{\pubdate}{\today}
\newcommand{\setpubdate}[1]{\renewcommand{\pubdate}{#1}}

\iflatexml
\usepackage{hyperref}
\else
%% then we are making a PDF, so include things that LaTeXML can't handle: 
%% docbook style, \RaggedRight
\usepackage{ifxetex}
\usepackage{pslatex} % fixes fonts; in particular sets a better-fitting \tt font
\usepackage[hyperlink]{asciidoc-dblatex} 
%\usepackage{verbatim}
\usepackage{ragged2e}
\setlength{\RaggedRightRightskip}{0pt plus 4em}
\RaggedRight
\renewcommand{\DBKpubdate}{\pubdate}
\renewcommand{\DBKreleaseinfo}{}
\fi

% set hypertext links to be dark blue:
\definecolor{darkblue}{rgb}{0,0,0.8}
\definecolor{sidebar}{rgb}{0.5,0.5,0.7}
\hypersetup{colorlinks=true,urlcolor=darkblue,linkcolor=darkblue}

%%%%%%%%%%%%%%%%%%%%%%%%%%%%%%%%%%%%%%%%%%%%%%%%%%%%%%%%%%%%%%%%%%%%%%%%%%%%%
%
% Define macros for handling javadoc class and method references
%
%%%%%%%%%%%%%%%%%%%%%%%%%%%%%%%%%%%%%%%%%%%%%%%%%%%%%%%%%%%%%%%%%%%%%%%%%%%%%
\makeatletter

% code inspired by http://stackoverflow.com/questions/2457780/latex-apply-an-operation-to-every-character-in-a-string
\def\removeargs #1{\doremoveargs#1$\wholeString\unskip}
\def\doremoveargs#1#2\wholeString{\if#1$%
\else\if#1({()}\else{#1}\taketherest#2\fi\fi}
\def\taketherest#1\fi
{\fi \doremoveargs#1\wholeString}

% Note: still doesn't work properly when called on macro output ...
% i.e., \dottoslash{\concatnames{model}{base}{foo}} fails 
\def\dottoslash #1{\dodottoslash#1$\wholeString\unskip}
\def\dodottoslash#1#2\wholeString{\if#1$%
\else\if#1.{/}\else{#1}\fi\dottaketherest#2\fi}
\def\dottaketherest#1\fi{\fi \dodottoslash#1\wholeString}

% concatenates up to three class/method names together, adding '.' characters
% between them. The first and/or second argument may be empty, in which case
% the '.' is omitted. To check to see if these arguments are empty, we
% use a contruction '\if#1@@', which will return true iff #1 is empty
% (on the assumption that #1 will not contain a '@' character).
\def\concatnames
#1#2#3{\if#1@@\if#2@@#3\else #2.#3\fi\else\if#2@@#1.#3\else#1.#2.#3\fi\fi}

\newcommand{\javabase}{}
\newcommand{\setjavabase}[1]{\renewcommand{\javabase}{#1}}
\iflatexml
\newcommand{\javaclassx}[2][]{%
% Includes code to prevent an extra '.' at the front if #1 is empty. It
% works like this: if '#1' is empty, then '#1.' expands to '.', and so 
% '\if#1..' will return true, in which case we just output '#2'.
\href{@JDOCBEGIN/\concatnames{\javabase}{#1}{#2}@JDOCEND}{#2}}
\newcommand{\javaclass}[2][]{%
\href{@JDOCBEGIN/\concatnames{}{#1}{#2}@JDOCEND}{#2}}

\newcommand{\javamethodArgsx}[2][]{%
\href{@JDOCBEGIN/\concatnames{\javabase}{#1}{#2}@JDOCEND}{#2}}
\newcommand{\javamethodArgs}[2][]{%
\href{@JDOCBEGIN/\concatnames{}{#1}{#2}@JDOCEND}{#2}}

\newcommand{\javamethodNoArgsx}[2][]{%
\href{@JDOCBEGIN/\concatnames{\javabase}{#1}{#2}@JDOCEND}{\removeargs{#2}}}
\newcommand{\javamethodNoArgs}[2][]{%
\href{@JDOCBEGIN/\concatnames{}{#1}{#2}@JDOCEND}{\removeargs{#2}}}
\else
\def\javaurl{http://www.artisynth.org/doc/javadocs/}
\newcommand{\javaclassx}[2][]{{\color{darkblue}#2}}
%\href{\javaurl\dottoslash{\concatnames{\javabase}{#1}{#2}}.html}{#2}}
\newcommand{\javamethodArgsx}[2][]{{\color{darkblue}#2}}
\newcommand{\javamethodNoArgsx}[2][]{{\color{darkblue}\removeargs{#2}}}
\newcommand{\javaclass}[2][]{{\color{darkblue}#2}}
%\href{\javaurl\dottoslash{\concatnames{\javabase}{#1}{#2}}.html}{#2}}
\newcommand{\javamethodArgs}[2][]{{\color{darkblue}#2}}
\newcommand{\javamethodNoArgs}[2][]{{\color{darkblue}\removeargs{#2}}}
\fi

\newcommand{\javamethod}{\@ifstar\javamethodNoArgs\javamethodArgs}
\newcommand{\javamethodx}{\@ifstar\javamethodNoArgsx\javamethodArgsx}

%%%%%%%%%%%%%%%%%%%%%%%%%%%%%%%%%%%%%%%%%%%%%%%%%%%%%%%%%%%%%%%%%%%%%%%%%%%%%
%
% Define macros for sidebars
%
%%%%%%%%%%%%%%%%%%%%%%%%%%%%%%%%%%%%%%%%%%%%%%%%%%%%%%%%%%%%%%%%%%%%%%%%%%%%%

\iflatexml
\newenvironment{sideblock}{\begin{quote}}{\end{quote}}
\else
\usepackage[strict]{changepage}
\definecolor{sidebarshade}{rgb}{1.0,0.97,0.8}
\newenvironment{sideblock}{%
    \def\FrameCommand{%
    \hspace{1pt}%
    {\color{sidebar}\vrule width 2pt}%
    %{\vrule width 2pt}%
    {\color{sidebarshade}\vrule width 4pt}%
    \colorbox{sidebarshade}%
  }%
  \MakeFramed{\advance\hsize-\width\FrameRestore}%
  \noindent\hspace{-4.55pt}% disable indenting first paragraph
  \begin{adjustwidth}{}{7pt}%
  %\vspace{2pt}\vspace{2pt}%
}
{%
  \vspace{2pt}\end{adjustwidth}\endMakeFramed%
}
\fi

\iflatexml
\newenvironment{shadedregion}{%
  \definecolor{shadecolor}{rgb}{0.96,0.96,0.98}%
  \begin{shaded*}%
% Put text inside a quote to create a surrounding blockquote that
% will properly accept the color and padding attributes
  \begin{quote}%
}
{%
  \end{quote}%
  \end{shaded*}%
}
\else
\newenvironment{shadedregion}{%
  \definecolor{shadecolor}{rgb}{0.96,0.96,0.98}%
  \begin{shaded*}%
}
{%
  \end{shaded*}%
}
\fi

% Wanted to create a 'listing' environment because lstlisting is
% tedious to type and because under latexml it may need
% some massaging to get it to work properly. But hard to do
% because of the verbatim nature of listing
%\iflatexml
%\newenvironment{listing}{\begin{lstlisting}}{\end{lstlisting}}%
%\else
%\newenvironment{listing}{\begin{lstlisting}}{\end{lstlisting}}%
%\fi

\iflatexml\else
% fancyhdr was complaining that it wanted a 36pt header height ...
\setlength{\headheight}{36pt}
\fi


% abbreviation for backslash character
\newcommand\BKS{\textbackslash}

\makeatother


\setcounter{tocdepth}{5}
\setcounter{secnumdepth}{3}

\title{Maspack Rendering Guide}
\author{Antonio S\'anchez}
\setpubdate{July 12, 2015}

\iflatexml
\date{}
\fi

% Listings settings
\definecolor{myblue}{rgb}{0,0,0.6}
\definecolor{mygreen}{rgb}{0,0.6,0}
\definecolor{mygray}{rgb}{0.5,0.5,0.5}
\definecolor{mylightgray}{rgb}{0.95,0.95,0.95}
\definecolor{mymauve}{rgb}{0.58,0,0.82}
\definecolor{myblack}{rgb}{0,0,0}
\lstset{
   language=Java,                   % text highlighting for Java
   breakatwhitespace=false,         % automatic breaks only at whitespace
   breaklines=true,                 % automatic line breaking
   commentstyle=\color{mygreen},    % comment style
   keepspaces=true,                 % keeps spaces in text
   keywordstyle=\color{myblue},     % keyword style
   numbers=none,                    % line-numbers; values: (none, left, right)
   numbersep=5pt,                   % how far the line-numbers are from code
   numberstyle=\tiny\color{mygray}, % line-numbers style
   showspaces=false,                % show spaces everywhere
   showstringspaces=false,          % underline spaces within strings
   showtabs=false,                  % show tabs
   stepnumber=1,                    % the step between two line-numbers
   stringstyle=\color{mymauve},     % string literal style
   tabsize=3,                       % sets default tabsize to 3 spaces
   backgroundcolor=\color{mylightgray}, % background color
   frame=single, 				         % adds a frame around the code
   rulesepcolor=\color{mygray},
   rulecolor=\color{myblack},
   framerule=0pt,
   xleftmargin=2.2ex,               % numbers inside box
   framexleftmargin=2.2ex,			   % indentation of frame
   escapeinside={(*@}{@*)},
}

\newenvironment{tightemize}{\begin{itemize}[nolistsep,noitemsep]}{\end{itemize}}
\newenvironment{enumertight}{\begin{enumerate}[nolistsep,noitemsep]}{\end{enumerate}}

\let\olditem\item
\newcommand{\lstitem}[2][]{\olditem[\texttt{#1}]\mbox{}\newline#2}
\newenvironment{lstdescription}{%
   \begin{description}[nolistsep]
   \let\item\lstitem
}{\end{description}}

\begin{document}

%\layout
\maketitle

\iflatexml{\large\pubdate}\fi

\tableofcontents

% basic links to other docs: http://www.artisynth.org/doc/html/xxx/xxx.html#sec

\section{Introduction}

This guide describes how to interface with the maspack renderers.  In most cases,
it should be sufficient to render your custom object by drawing a small collection of 
primitives using functions built-in to the renderer (see Section \ref{sec:primitives}).
For objects containing larger complex geometries, where rendering efficiency
is a concern, a rendering cache should be constructed in the form of a 
\javaclass{RenderObject} (Section \ref{sec:renderobject}).

\section{Primitive Objects \label{sec:primitives}}

The \javaclass{Renderer} interface provides functions for drawing simple
primitives and structures, including: points, lines, spheres, cylinders,
cones, spindles, arrows, tetrahedrons, pyramids, wedges, hexahedrons,
and axes.  \ldots

\section{Custom Render Objects}

For large or complex geometries, a customizable \javaclass{RenderObject} can be 
created to store a collection of vertices and primitives.  This can then be
passed to the renderer for drawing.  Internally, the renderer uses information
from the \lstinline{RenderObject} to set up a rendering cache.  This can greatly improve
rendering efficiency.

\subsection{The \texttt{RenderObject} class \label{sec:renderobject}}


The \lstinline{RenderObject} consists of:
\begin{tightemize}
   \item a single set of vertices, consisting of indices to particular attribute values
   \item one or more ``sets'' of positions
   \item zero or more ``sets'' of normals
   \item zero or more ``sets'' of colors
   \item zero or more ``sets'' of 2D texture coordinates
   \item zero more more ``groups'' of point primitives
   \item zero more more ``groups'' of line primitives
   \item zero more more ``groups'' of triangle primitives
\end{tightemize}
A vertex is treated as a 4-tuple of indices that refer to a particular position, normal (optional),
color (optional), and 2D texture coordinate (optional).  This allows for keeping different sized
lists for each of the attributes.  For example, a cube will have eight unique positions for the 
corners, and six unique normals for the faces.  
\begin{lstlisting}[caption=Adding attributes to make a cube,label=lst:cube:attrib]
RenderObject cube = new RenderObject();
// positions                                 // normals
int[] pIdxs = new int[8];                    int[] nIdxs = new int[6];
pIdxs[0] = cube.addPosition(-1f,-1f,-1f);    nIdxs[0] = cube.addNormal( 0f, 0f,-1f);
pIdxs[1] = cube.addPosition( 1f,-1f,-1f);    nIdxs[1] = cube.addNormal( 0f, 0f, 1f);
pIdxs[2] = cube.addPosition( 1f, 1f,-1f);    nIdxs[2] = cube.addNormal( 0f,-1f, 0f);
pIdxs[3] = cube.addPosition(-1f, 1f,-1f);    nIdxs[3] = cube.addNormal( 0f, 1f, 0f);
pIdxs[4] = cube.addPosition(-1f,-1f, 1f);    nIdxs[4] = cube.addNormal(-1f, 0f, 0f);
pIdxs[5] = cube.addPosition( 1f,-1f, 1f);    nIdxs[5] = cube.addNormal( 1f, 0f, 0f);
pIdxs[6] = cube.addPosition( 1f, 1f, 1f);
pIdxs[7] = cube.addPosition(-1f, 1f, 1f);
\end{lstlisting}
Each attribute entry is assigned a unique index per attribute, which can later be referred to 
when adding vertices.  Indices are increased sequentially, starting from zero.  In 
Listing \ref{lst:cube:attrib}, we have \lstinline{pIdxs[i]=i} and \lstinline{nIdxs[j]=j}.

A vertex can refer to at most one of each attribute.  This means that when building 
primitives, some vertex locations must be repeated in order to have different normals.
For example, for the corner of the cube at location (-1,-1,-1), there must be three
vertices defined, one each with normals (-1,0,0), (0,-1,0), and (0,0,-1).  Vertices
are defined using the indices from previously added attributes.  A vertex does not need
to have all attributes defined, although it is assumed that all vertices have a consistent
set of attributes (e.g.~either all vertices have a normal defined, or none do).  When 
adding a vertex, a negative index indicates that the
 attribute is not present.
\begin{lstlisting}[caption=Adding vertices to the cube,label=lst:cube:vertex]
// vertices
int[] vIdxs = new int[24];
// bottom    indices:(position, normal, color, texture)
vIdxs[0]  = cube.addVertex(0,0,-1,-1);  vIdxs[1]  = cube.addVertex(1,0,-1,-1);
vIdxs[2]  = cube.addVertex(2,0,-1,-1);  vIdxs[3]  = cube.addVertex(3,0,-1,-1);
// top
vIdxs[4]  = cube.addVertex(4,1,-1,-1);  vIdxs[5]  = cube.addVertex(5,1,-1,-1);
vIdxs[6]  = cube.addVertex(6,1,-1,-1);  vIdxs[7]  = cube.addVertex(7,1,-1,-1);
// left
vIdxs[8]  = cube.addVertex(0,2,-1,-1);  vIdxs[9]  = cube.addVertex(1,2,-1,-1);
vIdxs[10] = cube.addVertex(4,2,-1,-1);  vIdxs[11] = cube.addVertex(5,2,-1,-1);
// right
vIdxs[12] = cube.addVertex(2,3,-1,-1);  vIdxs[13] = cube.addVertex(3,3,-1,-1);
vIdxs[14] = cube.addVertex(6,3,-1,-1);  vIdxs[15] = cube.addVertex(7,3,-1,-1);
// front
vIdxs[16] = cube.addVertex(3,4,-1,-1);  vIdxs[17] = cube.addVertex(0,4,-1,-1);
vIdxs[18] = cube.addVertex(7,4,-1,-1);  vIdxs[19] = cube.addVertex(4,4,-1,-1);
// back
vIdxs[20] = cube.addVertex(1,5,-1,-1);  vIdxs[21] = cube.addVertex(2,5,-1,-1);
vIdxs[22] = cube.addVertex(5,5,-1,-1);  vIdxs[23] = cube.addVertex(6,5,-1,-1);
\end{lstlisting}
Each vertex is assigned a unique index, starting at zero and increasing sequentially.  In Listing
\ref{lst:cube:vertex}, we have \lstinline{vIdxs[i]=i}.  The cube has positions and normals
defined, but no vertex colors or texture coordinates.

Primitives can then be added using the vertex indices.  Three types of primitives are
available: points, lines, and triangles.  In Listing \ref{lst:cube:primitives}, we complete
the definition of the cube using triangles.
\begin{lstlisting}[caption={Adding triangle primitives to complete the cube}, label=lst:cube:primitives]
// triangular faces
cube.addTriangle( 2, 1, 0);  cube.addTriangle( 3, 2, 0);  // bottom
cube.addTriangle( 4, 5, 6);  cube.addTriangle( 7, 4, 6);  // top
cube.addTriangle( 8, 9,10);  cube.addTriangle( 9,11,10);  // left
cube.addTriangle(12,13,14);  cube.addTriangle(13,15,14);  // right
cube.addTriangle(16,17,18);  cube.addTriangle(17,19,18);  // front
cube.addTriangle(20,21,22);  cube.addTriangle(21,23,22);  // back
\end{lstlisting}

\subsubsection{Setting a ``current'' attribute}

The manual construction as above can be quite tedious for simple shapes such as a cube.
Rather than adding all attributes, keeping track of indices, and using attribute indices
to build vertices, a ``current'' attribute value can be defined to be used in all subsequent 
vertices.
\begin{lstlisting}[caption={Construction of a cube using current attributes},label=lst:cube:current]
RenderObject cube = new RenderObject();
int[] vIdxs = new int[24];
// bottom
cube.normal( 0f, 0f,-1f);  // normal for subsequent vertices
// vertices (adds position automatically)
vIdxs[0]  = cube.vertex(-1f,-1f,-1f);  vIdxs[1]  = cube.vertex( 1f,-1f,-1f);
vIdxs[2]  = cube.vertex( 1f, 1f,-1f);  vIdxs[3]  = cube.vertex(-1f, 1f,-1f);
// top
cube.normal( 0f, 0f, 1f);  // new normal for subsequent vertices
vIdxs[4]  = cube.vertex(-1f,-1f, 1f);  vIdxs[5]  = cube.vertex( 1f,-1f, 1f);
vIdxs[6]  = cube.vertex( 1f, 1f, 1f);  vIdxs[7]  = cube.vertex(-1f, 1f, 1f);
// left
cube.normal( 0f,-1f, 0f);
vIdxs[8]  = cube.vertex(-1f,-1f,-1f);  vIdxs[9]  = cube.vertex( 1f,-1f,-1f);
vIdxs[10] = cube.vertex(-1f,-1f, 1f);  vIdxs[11] = cube.vertex( 1f,-1f, 1f);
// right
cube.normal( 0f, 1f, 0f);
vIdxs[12] = cube.vertex( 1f, 1f,-1f);  vIdxs[13] = cube.vertex(-1f, 1f,-1f);
vIdxs[14] = cube.vertex( 1f, 1f, 1f);  vIdxs[15] = cube.vertex(-1f, 1f, 1f);
// front
cube.normal(-1f, 0f, 0f);
vIdxs[16] = cube.vertex(-1f, 1f,-1f);  vIdxs[17] = cube.vertex(-1f,-1f,-1f);
vIdxs[18] = cube.vertex(-1f, 1f, 1f);  vIdxs[19] = cube.vertex(-1f,-1f, 1f);
// back
cube.normal( 1f, 0f, 0f);
vIdxs[20] = cube.vertex( 1f,-1f,-1f);  vIdxs[21] = cube.vertex( 1f, 1f,-1f);
vIdxs[22] = cube.vertex( 1f,-1f, 1f);  vIdxs[23] = cube.vertex( 1f, 1f, 1f);
\end{lstlisting}
The above uses a short-hand for adding a position and creating a vertex simultaneously
with the command \lstinline{vertex(...)}. Color and texture attributes can be 
set in a similar way to the normals in Listing \ref{lst:cube:current}.  Triangle primitives are 
still constructed as before, using the appropriate vertex indices (Listing \ref{lst:cube:primitives}).
Note that in the above example, positions are duplicated for each vertex, so the
\lstinline{RenderObject} has an internal list of 24 positions rather than just
eight.  To reduce internal storage requirements, the two methods of adding
vertices can be combined.
\begin{lstlisting}[caption={Combining vertex construction methods},label=lst:cube:combined]
RenderObject cube = new RenderObject();
int[] vIdxs = new int[24];
// bottom
cube.normal( 0f, 0f,-1f);  // normal for subsequent vertices
// vertices (adds position automatically)
vIdxs[0]  = cube.vertex(-1f,-1f,-1f);  vIdxs[1]  = cube.vertex( 1f,-1f,-1f);
vIdxs[2]  = cube.vertex( 1f, 1f,-1f);  vIdxs[3]  = cube.vertex(-1f, 1f,-1f);
// top
cube.normal( 0f, 0f, 1f);  // new normal for subsequent vertices
vIdxs[4]  = cube.vertex(-1f,-1f, 1f);  vIdxs[5]  = cube.vertex( 1f,-1f, 1f);
vIdxs[6]  = cube.vertex( 1f, 1f, 1f);  vIdxs[7]  = cube.vertex(-1f, 1f, 1f);
// left
cube.normal( 0f,-1f, 0f);
// add vertex based on previously added position, with current normal
vIdxs[8]  = cube.addVertex(0);         vIdxs[9]  = cube.addVertex(1);
vIdxs[10] = cube.addVertex(4);         vIdxs[11] = cube.addVertex(5);
// right
cube.normal( 0f, 1f, 0f);
vIdxs[12] = cube.addVertex(2);         vIdxs[13] = cube.addVertex(3);
vIdxs[14] = cube.addVertex(6);         vIdxs[15] = cube.addVertex(7);
// front
cube.normal(-1f, 0f, 0f);
vIdxs[16] = cube.addVertex(3);         vIdxs[17] = cube.addVertex(0);
vIdxs[18] = cube.addVertex(7);         vIdxs[19] = cube.addVertex(4);
// back
cube.normal( 1f, 0f, 0f);
vIdxs[20] = cube.addVertex(1);         vIdxs[21] = cube.addVertex(2);
vIdxs[22] = cube.addVertex(5);         vIdxs[23] = cube.addVertex(6);
\end{lstlisting}
In Listing \ref{lst:cube:combined}, vertices 8--23 are constructed by referring to a supplied
position index, and using the current normal, color and texture coordinates.

\subsubsection{Adding primitives in ``build'' mode}

The \javaclass{RenderObject} can also be systematically constructed using a ``build mode'',
similar to drawing objects in legacy OpenGL.  This causes primitives to automatically be
generated as vertices are added.  The following modes are supported:
\begin{center}
\begin{tabular}{ll}
   \hline\hline
   BuildMode & Equivalent OpenGL Mode\\
   \hline
   BuildMode.POINTS & GL\_POINTS\\
   BuildMode.LINES & GL\_LINES\\
   BuildMode.LINE\_STRIP & GL\_LINE\_STRIP\\
   BuildMode.LINE\_LOOP & GL\_LINE\_LOOP\\
   BuildMode.TRIANGLES & GL\_TRIANGLES\\
   BuildMode.TRIANGLE\_STRIP & GL\_TRIANGLE\_STRIP\\
   BuildMode.TRIANGLE\_FAN & GL\_TRIANGLE\_FAN\\
   \hline
   \end{tabular}
\end{center}
Primitive construction begins with \javamethod{beginBuild(BuildMode)}, and 
ends with  \javamethod{endBuild()}. Primitives are only gauranteed to be valid after 
the \lstinline{endBuild()} call.
\begin{lstlisting}[caption={Building a cylinder}, label=lst:cylinder]
RenderObject cylinder = new RenderObject();
int nSlices = 64;
      
// reserve memory
cylinder.ensurePositionCapacity(2*nSlices);  // top and bottom ring
cylinder.ensureNormalCapacity(nSlices+2);    // sides and caps
cylinder.ensureVertexCapacity(4*nSlices);    // top/bottom sides, top/bottom caps
cylinder.ensureTriangleCapacity(2*nSlices+2*(nSlices-2));  // sides, caps
      
// sides
cylinder.beginBuild(BuildMode.TRIANGLE_STRIP);
for (int i=0; i<nSlices; i++) {
   double angle = 2*Math.PI/nSlices*i;
   float x = (float)Math.cos(angle);
   float y = (float)Math.sin(angle);
   cylinder.normal(x, y, 0);
   cylinder.vertex(x, y, 1);  // top
   cylinder.vertex(x, y, 0);  // bottom
}
cylinder.endBuild();

// connect ends around cylinder
cylinder.addTriangle(2*nSlices-2, 2*nSlices-1, 0);
cylinder.addTriangle(0, 2*nSlices-1, 1);
      
// caps
cylinder.beginBuild(BuildMode.TRIANGLE_FAN);
int nidx = cylinder.addNormal(0,0,1);
for (int i=0; i<nSlices; i++) {
   cylinder.addVertex(2*i, nidx);    // even positions (top)
}
cylinder.endBuild();

// bottom
cylinder.beginBuild(BuildMode.TRIANGLE_FAN);
int nidx = cylinder.addNormal(0,0,-1);
cylinder.addVertex(1, nidx);
for (int i=1; i<nSlices; i++) {
   int j = nSlices-i;
   cylinder.addVertex(2*j+1, nidx);  // odd positions (bottom)
}
cylinder.endBuild();
\end{lstlisting}
In Listing \ref{lst:cylinder}, we first reserve memory for the required attributes, vertices
and triangles.  This is not a required step, but does help with internal storage.  Here,
we use a triangle strip to construct the rounded sides of the cylinder, and triangle fans
to construct the caps.

\subsubsection{Static, dynamic and streaming}

By default, it is assumed that all attributes, vertices, and primitives are static: they
are constructed once, used many times, and never modified.  A static vertex attribute
cannot be modified after either:
\begin{tightemize}
   \item the \lstinline{RenderObject} data is ``committed'' by manually calling \javamethod{commit()}
   \item the \lstinline{RenderObject} is first used by a renderer (which will call \lstinline{commit()})
\end{tightemize}
Vertex attributes can be made `dynamic' using the method \lstinline{set<Attribute>Dynamic(true)}, 
where \lstinline{<Attribute>} is one of \lstinline{Positions}, \lstinline{Normals}, 
\lstinline{Colors}, or \lstinline{TextureCoords}.  A dynamic attribute \emph{can} be 
modified after use.  Values are updated with \lstinline{set<Attribute>(int idx, ...)}.
Note that no new attribute can be added once the vertex data is committed; only dynamic
attribute values can be modified.  Vertices and primitives are always considered 
static.  
\begin{table}
\centering
\caption{Vertex attribute usage modes}
\begin{tabular}{ll}
\hline\hline
Mode & Usage\\
\hline
Static & Constructed once, never modified, used many times\\
Dynamic & Constructed once, modified many times, used many times\\
Streaming & Constructed once, never modified, used a few times\\
\hline
\end{tabular}
\end{table}

If it is known that the \lstinline{RenderObject} will only be used a few times before being discarded,
then it should be set to `streaming' mode.  This provides a hint to renderers so it can optimize
graphics memory.  An object can be set as streaming using \javamethod{setStreaming(boolean)}.

\subsubsection{Detecting changes}

To facilitate detection of changes to the \lstinline{RenderObject} when used by multiple
renderers, a `version' system is implemented.  If the object has been modified or updated,
then the function \javamethod{getVersion()} will return a new version number.  
Renderers keep track of the last version of the object they have observed, and if the version
number has increased, they know to query for updated values.

Each vertex attribute, the vertices themselves, and each primitive type has its own version
number, so renderers can determine what type of information has changed.  Individual changes
or changesets are not tracked.

\subsubsection{Multiple attribute sets}

Vertices refer to attributes by index, which point to a particular attribute value in a list.
It may be desireable to replace the entire list to change some rendering effect
on-the-fly.  For example, when rendering a triangular surface mesh, one may want the ability
to draw the mesh both with a smooth shading, and with a flat shading.  The positions, colors
and textures are identical, but the normals need to be switched to use the flat face normals
to render in `flat' mode.  Another example is with a transform widget, where we may wish to 
highlight a particular axis in a different color on mouse-over.  To support
this, we introduce the concept of multiple `sets' of an attribute.

A vertex attribute `set' is defined as a potential list of the attribute values to pull from
when drawing the object.  For each attribute, there is a currently `active' set which is
used when either modifying values using \lstinline{set<Attribute>(int idx,...)}, or queried
with \lstinline{get<Attribute>(int idx)}.  By default, the \lstinline{RenderObject} has zero 
sets of each attribute.  When an attribute value is added, such as with \lstinline{addPosition(...)},
an initial set of the attribute is automatically created.  A new set for a particular
attribute can be manually created using \lstinline{create<Attribute>Set()} or 
\lstinline{create<Attribute>SetFrom(int)}.  The first method duplicates the currently
`active' set, whereas the latter duplicates another set specified by index.
Both these functions return an index to be used to refer to the newly created
attribute set.  The new set immediately becomes active.  The active set can be changed using
\lstinline{<attribute>Set(int)}.

\begin{lstlisting}[caption={3D Transform dragger with multiple color sets},label=lst:dragger:multicolor]
final int QUARTER_CIRCLE_RESOLUTION = 32;
final int FULL_CIRCLE_RESOLUTION = 4*QUARTER_CIRCLE_RESOLUTION;
final float TRANS_BOX_SIZE = 0.4f;
final float TRANS_ROT_SIZE = 0.8f;
      
RenderObject transrotr = new RenderObject();
      
// 3 axis, 3 plane boxes, 3 rotation quarter circles
int xcolor  = transrotr.addColor(1.0f, 0.0f, 0.0f, 1.0f);
int ycolor  = transrotr.addColor(0.0f, 1.0f, 0.0f, 1.0f);
int zcolor  = transrotr.addColor(0.0f, 0.0f, 1.0f, 1.0f);
int yzcolor = transrotr.addColor(0.5f, 0.5f, 0.5f, 1.0f);
int zxcolor = transrotr.addColor(0.5f, 0.5f, 0.5f, 1.0f);
int xycolor = transrotr.addColor(0.5f, 0.5f, 0.5f, 1.0f);
int xrcolor = transrotr.addColor(1.0f, 0.0f, 0.0f, 1.0f);
int yrcolor = transrotr.addColor(0.0f, 1.0f, 0.0f, 1.0f);
int zrcolor = transrotr.addColor(0.0f, 0.0f, 1.0f, 1.0f);
      
// 9 more color sets, one each with a component highlighted yellow
int[] colors = new int[]{xcolor,  ycolor,  zcolor,
                         yzcolor, zxcolor, xycolor,
                         xrcolor, yrcolor, zrcolor};
for (int i=0; i<colors.length; i++) {
   transrotr.createColorSetFrom(0);                // duplicate original color set
   transrotr.setColor(i, 1.0f, 1.0f, 0.0f, 1.0f);  // replace ith color with yellow
}
      
// axes
transrotr.beginBuild(BuildMode.LINES);
transrotr.color(xcolor);  transrotr.vertex(0, 0, 0);  transrotr.vertex(1, 0, 0);
transrotr.color(ycolor);  transrotr.vertex(0, 0, 0);  transrotr.vertex(0, 1, 0);
transrotr.color(zcolor);  transrotr.vertex(0, 0, 0);  transrotr.vertex(0, 0, 1);
transrotr.endBuild();
      
// yz-plane
int v0, v1, v2;
transrotr.color(yzcolor);
v0 = transrotr.vertex(0, TRANS_BOX_SIZE, 0);
v1 = transrotr.vertex(0, TRANS_BOX_SIZE, TRANS_BOX_SIZE);
v2 = transrotr.vertex(0, 0, TRANS_BOX_SIZE);
transrotr.addLineStrip(v0, v1, v2);

// zx-plane
transrotr.color(zxcolor);
v0 = transrotr.vertex(0, 0, TRANS_BOX_SIZE);
v1 = transrotr.vertex(TRANS_BOX_SIZE, 0, TRANS_BOX_SIZE);
v2 = transrotr.vertex(TRANS_BOX_SIZE, 0, 0);
transrotr.addLineStrip(v0, v1, v2);
      
// xy-plane
transrotr.color(xycolor);
v0 = transrotr.vertex(TRANS_BOX_SIZE, 0, 0);
v1 = transrotr.vertex(TRANS_BOX_SIZE, TRANS_BOX_SIZE, 0);
v2 = transrotr.vertex(0, TRANS_BOX_SIZE, 0);
transrotr.addLineStrip(v0, v1, v2);
      
// x-rotation
transrotr.beginBuild(BuildMode.LINE_STRIP);
transrotr.color(xrcolor);
for (int i = 0; i <= QUARTER_CIRCLE_RESOLUTION; i++) {
   double ang = 2 * Math.PI * i / (FULL_CIRCLE_RESOLUTION);
   transrotr.vertex(
      0f, TRANS_ROT_SIZE*(float)Math.cos(ang), TRANS_ROT_SIZE*(float)Math.sin(ang));
}
transrotr.endBuild();
      
// y-rotation
transrotr.beginBuild(BuildMode.LINE_STRIP);
transrotr.color(yrcolor);
for (int i = 0; i <= QUARTER_CIRCLE_RESOLUTION; i++) {
   double ang = 2 * Math.PI * i / (FULL_CIRCLE_RESOLUTION);
   transrotr.vertex(
      TRANS_ROT_SIZE*(float)Math.cos(ang), 0f, TRANS_ROT_SIZE*(float)Math.sin(ang));
}
transrotr.endBuild();
      
// z-rotation
transrotr.beginBuild(BuildMode.LINE_STRIP);
transrotr.color(zrcolor);
for (int i = 0; i <= QUARTER_CIRCLE_RESOLUTION; i++) {
   double ang = 2 * Math.PI * i / (FULL_CIRCLE_RESOLUTION);
   transrotr.vertex( 
      TRANS_ROT_SIZE*(float)Math.cos(ang), TRANS_ROT_SIZE*(float)Math.sin(ang), 0f);
}
transrotr.endBuild();
\end{lstlisting}

In Listing \ref{lst:dragger:multicolor}, a 3D translator-rotator is constructed with ten
total sets of colors.  The zeroeth set contains the default colors.  Sets one through
nine each have one component of the widget highlighted in yellow.  To draw the object
with a particular component highlighted, set the appropriate color set to be active
before passing it to the renderer.  For example, to highlight rotation around the
x-axis, use \lstinline{transrotr.colorSet(7)}.

Note that there is only one set of vertices, so care must be taken to ensure
that indexing is consistent between multiple attribute sets.  That is, all position
attribute sets must have the same size, all normal attribute sets must have the same
size, etc\ldots.  This is enforced by adding an attribute to all existing sets whenever 
\lstinline{add<Attribute>(...)} is called.  To customize values between attribute
sets, use \lstinline{set<Attribute(int idx,...)} to modify the values.

\subsubsection{Multiple primitive groups}

The \javaclass{RenderObject} can also have multiple groups of a particular primitive type.  This
is to allow for separate draw calls to render different parts of the object.  For example, consider
a triangular surface mesh consisting of a million faces that is to be drawn with the left half red, 
and the right half yellow.  One way to accomplish this is to add a vertex color attribute to each vertex.  
This will end up being quite memory inefficient, since the renderer will need to duplicate the color
for every vertex in the vertex buffer.  The alternative is to create two distinct triangle groups
and draw the mesh with two draw calls, changing the global color between them.  A particular primitive
group can be made active using the method \lstinline{<primitive>Group(int gidx)}, 
where \lstinline{<primitive>} is one of \lstinline{point}, \lstinline{line}, or \lstinline{triangle}.
There is no length consistency requirements between groups of a particular primitive type.

\subsection{Drawing \texttt{RenderObject}s}

\lstinline{RenderObject}s can be passed directly to the renderer for drawing.  Several methods for
drawing exist:
\begin{lstdescription}
   \item[Renderer.draw(RenderObject robj)] draws the current active groups of points, lines, and
   triangles (if present).
   \item[Renderer.drawPoints(RenderObject robj)] draws the current active point group
   \item[Renderer.drawPoints(RenderObject robj, PointStyle style, float r)] draws the current 
   active point group using a particular style (point, sphere) and radius (pixel width, sphere radius)
   \item[Renderer.drawLines(RenderObject robj)] draws the current active line group
   \item[Renderer.drawLines(RenderObject robj, LineStyle style, float r)] draws the current 
   active line group using a particular style (line, cylinder, spindle, arrow) and radius
   (pixel width, cylinder/ellipsoid/arrow radius)
   \item[Renderer.drawTriangles(RenderObject robj)] draws the current active triangle group
   \item[Renderer.drawVertices(RenderObject robj, VertexDrawMode mode)] draws the complete
      list of vertices using a particular \lstinline{mode}.  Currently supported modes
      include:\\
      \lstinline{VertexDrawMode.POINTS}, \lstinline{VertexDrawMode.LINES}, \lstinline{VertexDrawMode.LINE_STRIP},\\
      \lstinline{VertexDrawMode.LINE_LOOP}, \lstinline{VertexDrawMode.TRIANGLES}, \lstinline{VertexDrawMode.TRIANGLE_STRIP},\\
      \lstinline{VertexDrawMode.TRIANGLE_FAN}.
\end{lstdescription}



% \iflatexml
% \bibliographystyle{plain}
% \bibliography{references}
% \else
% \section*{References}
% \bibliographystyle{plain}
% \begin{btSect}{references}
% \btPrintCited
% \end{btSect}
% \fi

\end{document}
