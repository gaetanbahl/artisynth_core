\documentclass{article}
%%
%% Default settings for artisynth
%%
\NeedsTeXFormat{LaTeX2e}
%%\ProvidesPackage{artisynthDoc}[2012/04/05]

\usepackage[T1]{fontenc}
\usepackage[latin1]{inputenc}
\usepackage{listings}
\usepackage{makeidx}
\usepackage{latexml}
\usepackage{graphicx}
\usepackage{framed}
\usepackage{color}

\newcommand{\pubdate}{\today}
\newcommand{\setpubdate}[1]{\renewcommand{\pubdate}{#1}}

\iflatexml
\usepackage{hyperref}
\else
%% then we are making a PDF, so include things that LaTeXML can't handle: 
%% docbook style, \RaggedRight
\usepackage{ifxetex}
\usepackage{pslatex} % fixes fonts; in particular sets a better-fitting \tt font
\usepackage[hyperlink]{asciidoc-dblatex} 
%\usepackage{verbatim}
\usepackage{ragged2e}
\setlength{\RaggedRightRightskip}{0pt plus 4em}
\RaggedRight
\renewcommand{\DBKpubdate}{\pubdate}
\renewcommand{\DBKreleaseinfo}{}
\fi

% set hypertext links to be dark blue:
\definecolor{darkblue}{rgb}{0,0,0.8}
\definecolor{sidebar}{rgb}{0.5,0.5,0.7}
\hypersetup{colorlinks=true,urlcolor=darkblue,linkcolor=darkblue}

%%%%%%%%%%%%%%%%%%%%%%%%%%%%%%%%%%%%%%%%%%%%%%%%%%%%%%%%%%%%%%%%%%%%%%%%%%%%%
%
% Define macros for handling javadoc class and method references
%
%%%%%%%%%%%%%%%%%%%%%%%%%%%%%%%%%%%%%%%%%%%%%%%%%%%%%%%%%%%%%%%%%%%%%%%%%%%%%
\makeatletter

% code inspired by http://stackoverflow.com/questions/2457780/latex-apply-an-operation-to-every-character-in-a-string
\def\removeargs #1{\doremoveargs#1$\wholeString\unskip}
\def\doremoveargs#1#2\wholeString{\if#1$%
\else\if#1({()}\else{#1}\taketherest#2\fi\fi}
\def\taketherest#1\fi
{\fi \doremoveargs#1\wholeString}

% Note: still doesn't work properly when called on macro output ...
% i.e., \dottoslash{\concatnames{model}{base}{foo}} fails 
\def\dottoslash #1{\dodottoslash#1$\wholeString\unskip}
\def\dodottoslash#1#2\wholeString{\if#1$%
\else\if#1.{/}\else{#1}\fi\dottaketherest#2\fi}
\def\dottaketherest#1\fi{\fi \dodottoslash#1\wholeString}

% concatenates up to three class/method names together, adding '.' characters
% between them. The first and/or second argument may be empty, in which case
% the '.' is omitted. To check to see if these arguments are empty, we
% use a contruction '\if#1@@', which will return true iff #1 is empty
% (on the assumption that #1 will not contain a '@' character).
\def\concatnames
#1#2#3{\if#1@@\if#2@@#3\else #2.#3\fi\else\if#2@@#1.#3\else#1.#2.#3\fi\fi}

\newcommand{\javabase}{}
\newcommand{\setjavabase}[1]{\renewcommand{\javabase}{#1}}
\iflatexml
\newcommand{\javaclassx}[2][]{%
% Includes code to prevent an extra '.' at the front if #1 is empty. It
% works like this: if '#1' is empty, then '#1.' expands to '.', and so 
% '\if#1..' will return true, in which case we just output '#2'.
\href{@JDOCBEGIN/\concatnames{\javabase}{#1}{#2}@JDOCEND}{#2}}
\newcommand{\javaclass}[2][]{%
\href{@JDOCBEGIN/\concatnames{}{#1}{#2}@JDOCEND}{#2}}

\newcommand{\javamethodArgsx}[2][]{%
\href{@JDOCBEGIN/\concatnames{\javabase}{#1}{#2}@JDOCEND}{#2}}
\newcommand{\javamethodArgs}[2][]{%
\href{@JDOCBEGIN/\concatnames{}{#1}{#2}@JDOCEND}{#2}}

\newcommand{\javamethodNoArgsx}[2][]{%
\href{@JDOCBEGIN/\concatnames{\javabase}{#1}{#2}@JDOCEND}{\removeargs{#2}}}
\newcommand{\javamethodNoArgs}[2][]{%
\href{@JDOCBEGIN/\concatnames{}{#1}{#2}@JDOCEND}{\removeargs{#2}}}
\else
\def\javaurl{http://www.artisynth.org/doc/javadocs/}
\newcommand{\javaclassx}[2][]{{\color{darkblue}#2}}
%\href{\javaurl\dottoslash{\concatnames{\javabase}{#1}{#2}}.html}{#2}}
\newcommand{\javamethodArgsx}[2][]{{\color{darkblue}#2}}
\newcommand{\javamethodNoArgsx}[2][]{{\color{darkblue}\removeargs{#2}}}
\newcommand{\javaclass}[2][]{{\color{darkblue}#2}}
%\href{\javaurl\dottoslash{\concatnames{\javabase}{#1}{#2}}.html}{#2}}
\newcommand{\javamethodArgs}[2][]{{\color{darkblue}#2}}
\newcommand{\javamethodNoArgs}[2][]{{\color{darkblue}\removeargs{#2}}}
\fi

\newcommand{\javamethod}{\@ifstar\javamethodNoArgs\javamethodArgs}
\newcommand{\javamethodx}{\@ifstar\javamethodNoArgsx\javamethodArgsx}

%%%%%%%%%%%%%%%%%%%%%%%%%%%%%%%%%%%%%%%%%%%%%%%%%%%%%%%%%%%%%%%%%%%%%%%%%%%%%
%
% Define macros for sidebars
%
%%%%%%%%%%%%%%%%%%%%%%%%%%%%%%%%%%%%%%%%%%%%%%%%%%%%%%%%%%%%%%%%%%%%%%%%%%%%%

\iflatexml
\newenvironment{sideblock}{\begin{quote}}{\end{quote}}
\else
\usepackage[strict]{changepage}
\definecolor{sidebarshade}{rgb}{1.0,0.97,0.8}
\newenvironment{sideblock}{%
    \def\FrameCommand{%
    \hspace{1pt}%
    {\color{sidebar}\vrule width 2pt}%
    %{\vrule width 2pt}%
    {\color{sidebarshade}\vrule width 4pt}%
    \colorbox{sidebarshade}%
  }%
  \MakeFramed{\advance\hsize-\width\FrameRestore}%
  \noindent\hspace{-4.55pt}% disable indenting first paragraph
  \begin{adjustwidth}{}{7pt}%
  %\vspace{2pt}\vspace{2pt}%
}
{%
  \vspace{2pt}\end{adjustwidth}\endMakeFramed%
}
\fi

\iflatexml
\newenvironment{shadedregion}{%
  \definecolor{shadecolor}{rgb}{0.96,0.96,0.98}%
  \begin{shaded*}%
% Put text inside a quote to create a surrounding blockquote that
% will properly accept the color and padding attributes
  \begin{quote}%
}
{%
  \end{quote}%
  \end{shaded*}%
}
\else
\newenvironment{shadedregion}{%
  \definecolor{shadecolor}{rgb}{0.96,0.96,0.98}%
  \begin{shaded*}%
}
{%
  \end{shaded*}%
}
\fi

% Wanted to create a 'listing' environment because lstlisting is
% tedious to type and because under latexml it may need
% some massaging to get it to work properly. But hard to do
% because of the verbatim nature of listing
%\iflatexml
%\newenvironment{listing}{\begin{lstlisting}}{\end{lstlisting}}%
%\else
%\newenvironment{listing}{\begin{lstlisting}}{\end{lstlisting}}%
%\fi

\iflatexml\else
% fancyhdr was complaining that it wanted a 36pt header height ...
\setlength{\headheight}{36pt}
\fi


% abbreviation for backslash character
\newcommand\BKS{\textbackslash}

\makeatother


\setcounter{tocdepth}{5}
\setcounter{secnumdepth}{3}

\title{ArtiSynth Installation Guide}
\author{John Lloyd}
\iflatexml
\date{}
\fi

\begin{document}

\maketitle

\iflatexml{\large\today}\fi

\tableofcontents

\section{Introduction}

This document describes how to install and run ArtiSynth on your
computer system. There are typically two classes of ArtiSynth user:
{\it end users} who mainly want to run the program with existing models,
and {\it developers} who want to create their own models and/or modify the
ArtiSynth code base. End users should read through
Section \ref{RunningArtisynthSec}, while developers should read further.

\section{Supported Systems}

ArtiSynth is currently supported for the following systems:

\begin{itemize}

\item Windows (32 and 64 bit)
\item Linux (32 and 64 bit)
\item MacOS 10.6 (i.e., Snow Leopard), on 64 bit Intel platforms

\end{itemize}

\section{External Software Requirements}
\label{externalSoftwareSec}

The following external software packages are required
to run ArtiSynth:

\begin{description}

\item[Java JDK 1.6]\mbox{}

This goes by various other names, including Java
SE 6 and Java SDK 1.6. If necessary, it can be obtained for Windows
and Linux from
\href{http://java.sun.com/javase/downloads}{java.sun.com/javase/downloads},
and for MacOS from
\href{http://www.apple.com/java}{www.apple.com/java}. The full Java
development kit (JDK) is required, which comes with the Java compiler
{\tt javac}. The run time environment (JRE) will not be
sufficient. There is no need for extra bundles such at JavaFX,
NetBeans, or EE.

%Pardiso:: 
%The direct sparse solver Pardiso, by Olaf Schenk and his
%colleagues at the University of Basel, is required by many ArtiSynth
%applications. A copy of Pardiso 4.1 is bundled with the distribution,
%but users must obtain a licence in order to run it. The licence is
%free for noncommerical applications, and can be obtained quickly by
%filling out a web form. See Section \ref{PardisoLicenceSec} for details.

\item[OpenGL and JOGL (should already be installed)]\mbox{}

OpenGL is supplied by most operating systems, so there should be no
need to install it. JOGL is an interface that allows OpenGL to be
called from Java. The necessary JOGL libraries are included with the
ArtiSynth distribution. The version of JOGL currently used by
ArtiSynth is described in the file {\tt
\$ARTISYNTH\_HOME/lib/JOGL\_VERSION}.

%However, if you have another version
%of JOGL installed, then care may be
%required to avoid clashes between versions. See Section \ref{JOGLClashesSec}.

\end{description}

%The following other packages are recommended for users doing
%development work with ArtiSynth:

\section{Downloading and Installing}
\label{DownloadingSec}

ArtiSynth may be obtained either by downloading one of the packaged
distributions, or by checking out the most recent development version.
The latter is recommended for developers working
closely with the ArtiSynth team and is described in
Section \ref{obtainingDevelopmentVersionSec}.

To obtain one of the packaged distributions:

\begin{enumerate}

\item Go to \href{http://www.artisynth.org}{http://www.artisynth.org}

\item Select {\sf Software/Downloads}. This will bring up a page directing you 
to the {\sf ArtiSynth Download Form}.

\item Select the link to the {\sf ArtiSynth Download Form}.

\item Fill in and submit this form. An email containing a link and a password 
will automatically be sent to the e-mail address you provide in this form.

\item Open the e-mail mentioned above and click on the given link to find a 
password protected webpage.

\item Enter the password provided by the email. Take care that no spaces are 
included, as the system is sensitive to them.

\item This will lead to a different webpage, with links to various ArtiSynth 
distributions. Select any one to download, and click it.

\item Download the distribution, and unzip it in an appropriate
location on your computer.

\end{enumerate}

In this document, the location of the ArtiSynth installation directory
will be referred to by the variable {\tt ARTISYNTH\_HOME}.  Expansions
of this will be indicated by {\tt \$ARTISYNTH\_HOME}, so that if the
installation directory is {\tt /home/robert/project/artisynth\_2\_0},
then {\tt \$ARTISYNTH\_HOME/lib/Linux} will denote the directory
\begin{verbatim}
  /home/robert/project/artisynth_2_0/lib/Linux
\end{verbatim}

\begin{sideblock}
{\bf Important:}\\
On Windows, make sure ArtiSynth is not installed in a location
for which any of the folder names contain spaces
(i.e., {\tt Program Files}).
This is very important for ArtiSynth to run correctly.
\end{sideblock}

\section{Running ArtiSynth}
\label{RunningArtisynthSec}

This section describes how to start ArtiSynth for running
existing models and demos. 

\subsection{Windows}

The most direct way to start ArtiSynth is to run the batch file
{\tt artisynth.bat}, located in {\tt \$ARTISYNTH\_HOME\BKS bin}.  This can be
done by double-clicking on it in a file browser.

You can also create a shortcut to this batch file (by right clicking
on it and selecting {\sf Create Shortcut}), and then placing this shortcut
in either the {\sf START} menu or on the Desktop. However, the batch file
itself must remain in {\tt \$ARTISYNTH\_HOME\BKS bin}.

Alternatively, you can open a command interpreter ({\tt CMD}), navigate
to {\tt \$ARTISYNTH\_HOME}, and enter the command

\begin{verbatim}
 % bin\artisynth
\end{verbatim}

\begin{sideblock}
{\bf Note:}\\
If Java is installed in a non-conventional location, the system may
not be able to find it and this will cause {\tt artisynth.bat} to fail.
To fix this, locate the folder in which {\tt java.exe} resides, and let
the location of this folder be called JAVA\_DIR. You should then add
JAVA\_DIR to your system's {\tt PATH} environment variable (see
Section \ref{settingWindowsPathSec}).  
\end{sideblock}

%Alternatively, you can add JAVA\_DIR to
%the {\tt PATH} variable within {\tt artisynth.bat}:
%
%. Open {\tt artisynth.bat} using a text editor such as {\tt notepad} or {\tt wordpad};
%. Update the setting for {\tt JAVA\_DIR} in line three to its correct value;
%. Uncomment lines three and four by removing the leading "@REM".

\subsection{MacOS, Linux, and Windows running Cygwin}

%On MacOS, you can run ArtiSynth by double clicking on
%{\tt \$ARTISYNTH\_HOME/bin/artisynth.command} in a file browser.

On MacOS, Linux, and Windows systems running Cygwin (see
Section \ref{CygwinInstallationSec}), you can run ArtiSynth by opening a terminal
window, changing to the directory {\tt ARTISYNTH\_HOME}, and running the
command

\begin{verbatim}
 > bin/artisynth
\end{verbatim}

If you set your {\tt PATH} environment variable to include {\tt
\$ARTISYNTH\_HOME/bin} (see Section \ref{environmentVariablesSec}), you
can then run ArtiSynth using the command

\begin{verbatim}
 > artisynth
\end{verbatim}

from any directory.

On MacOS, you can also run ArtiSynth from a file browser by double
clicking on
\begin{verbatim}
  $ARTISYNTH_HOME/bin/artisynth.command
\end{verbatim}
({\tt artisynth.command} is just a
copy of {\tt artisynth}; the {\tt .command} suffix makes it
recognizable to the MacOS GUI as a command).
MacOS GUI).

{\tt artisynth} can be called with a number of options; to see these, run

\begin{verbatim}
 > artisynth -help
\end{verbatim}

\begin{sideblock}
The {\tt artisynth} command (and the {\tt artisynth.bat} command used by
Windows) are actually just convenience scripts that parse
options, set environment
variables and invoke Java using the ArtiSynth driver class
{\tt artisynth.core.driver.Launcher}. If you have independently set the
environment variables described in Section \ref{environmentVariablesSec},
then you can invoke ArtiSynth from Java directly using
%
\begin{verbatim}
 > java artisynth.core.driver.Launcher
\end{verbatim}
%
although some of the options provided by {\tt artisynth} would be unavailable.
\end{sideblock}

%\section{Obtaining a Pardiso Licence}
%\label{PardisoLicenceSec}
%
%Pardiso is free for non-commercial purposes but users must obtain a
%license key by registering on the Pardiso project website. This key is
%then placed in a file which Pardiso checks upon activation.
%
%. Go to the Pardiso website: http://www.pardiso-project.org
%. Click "Download Software and License"
%. Select `Academic'
%. A registration form will appear. Confirm acceptance of the license agreement 
%and fill in all required fields.
%. Submit the registration form and then go to your email.
%. Take the LICENSE KEY emailed to you and paste it in to a text file  
%(use a text editor, *not* a word processor. Word processors (like MS Word or
%Open Office) add additional characters. Examples of text editors are Wordpad
%(for Windows) or gedit or emacs (for Linux).
%. Save the file in your home directory as pardiso.lic.
%On Mac and Linux systems, your home directory is indicated by the {\tt HOME} 
%environment variable. On Windows, it is 
%{\tt C:/Documents and Settings/{user name}}.
%
%[IMPORTANT]
%========================================================
%  - Take particular care in filling in the fields {\tt User name} and
%    {\tt Host name}, since these must match the information on your
%    machine in order for the license key to work (See
%    Section \ref{userAndHostNamesSec}).
%
%  - The information is case sensitive and the licensing system cannot
%    handle foreign characters.
%
%  - Make sure to exclude whitespace, particularly in the Pardiso licence. 	
%    Don't leave spaces or unused lines.
%
%  - On Windows systems, host name information should be entered in
%    upper case, while the user name should be entered as it is.
%========================================================

\section{Configuring for Development Work}
\label{SettingsForDevelopmentSec}

Users intending to use ArtiSynth to develop models of their own, or to
modify or add to the ArtiSynth code base, will need to write and
compile Java code. This can either be done directly, using text
editors and the Java compiler {\tt javac} to create, modify and compile
the necessary code, or by using an integrated development
environment (IDE), such as Eclipse.  Directions on setting up the
Eclipse IDE are given in Section \ref{EclipseSec}.

ArtiSynth uses and references a number of environment variables, as
described in Section \ref{environmentVariablesSec}. It is recommended that
developers set these in their general system environment.  Although
the {\tt artisynth} command ({\tt artisynth.bat} on Windows) will try to set
these variables automatically if they are unset, setting them in the
general environment gives the developer access to a broader range of
commands in {\tt \$ARTISYNTH\_HOME/bin}, and permits the use of various IDEs
without having to set variables in each.

For development work on Windows, it is recommended to install the Unix
emulator Cygwin in order to allow use of the script-based commands in
{\tt \$ARTISYNTH\_HOME/bin}, as well as the {\tt Makefile} commands.  Cygwin
provides both the {\tt bash} and {\tt tcsh} shell environments, for which
environment variables can be set as described in Section \ref{bashEnvSec} or
Section \ref{tcshEnvSec}.  See Section \ref{CygwinInstallationSec} for instructions on
installation.

\subsection{Obtaining the latest development version}
\label{obtainingDevelopmentVersionSec}

If you are working closely with ArtiSynth or any of its associated
projects, you are encouraged to use the latest development version,
instead of the packaged releases. This will provide immediate access
to updates and bug fixes. In addition, if you have repository write
access, you will be able to check in code so that it can be kept
consistent with modifications to the ArtiSynth API.

At present, the development version is contained in a CVS repository,
and checking out a version requires obtaining a guest computer account
in the UBC department of Electrical and Computer Engineering
(ECE). This also provides repository write access. For information on
obtaining such an account, please send email to {\tt artisynth} at
{\tt ece.ubc.ca}.

When you have an ECE account, you can obtain the latest development
version by using CVS. This will require that CVS is installed on your
system. It should be installed by default on MacOS, and can be
obtained for Windows as part of Cygwin (see Section \ref{CygwinInstallationSec}).  For Ubuntu
or Debian style Linux systems, it can be acquired if needed by issuing
the following command from a terminal window:

\begin{verbatim}
 > sudo apt-get install cvs
\end{verbatim}

The latest development version of ArtiSynth can then be obtained by
issuing the following CVS command from a terminal window:

\begin{verbatim}
 > cvs -d :ext:userName@ssh.ece.ubc.ca:/ubc/ece/home/hct/other/hct/cvsroot checkout -P artisynth_2_0
\end{verbatim}

where {\tt userName} is the name of your ECE account. This will check
out the latest version into a directory called {\tt artisynth\_2\_0} located
in the directory from which the {\tt cvs} command was issued; the
{\tt artisynth\_2\_0} directory will now become {\tt ARTISYNTH\_HOME}.

Once you have checked out the latest version, you will also need to
download the {\tt .jar} files from the ArtiSynth server so that you
can compile the code. This can be done by executing the
command {\tt updateArtisynthLibs} in the directory {\tt
\$ARTISYNTH\_HOME/bin}:

\begin{verbatim}
 > cd $ARTISYNTH_HOME/bin
 > updateArtisynthLibs
\end{verbatim}

For more information on downloading library files, including
information on how to do this on Windows, see Section
\ref{LibUpdateSec}.

The latest development version of ArtiSynth is stored in a module that
is called {\tt artisynth\_2\_0} for historical reasons. However, this
name does not reflect the actual version number, which is stored
in the file {\tt VERSION}. Once you have checked out {\tt artisynth\_2\_0},
you can rename the top level directory to anything you like.

Given a checked out version of ArtiSynth, you can update it by issuing
the command

\begin{verbatim}
 > cvs update -dP
\end{verbatim}

within the {\tt ARTISYNTH\_HOME} directory. The options {\tt -dP} tell CVS to
include new directories that have been added and ignore (prune)
directories that have been removed.

It is also possible to check out ArtiSynth within the Eclipse IDE; see
Section \ref{EclipseSec}.

\begin{sideblock}
The latest development version of ArtiSynth is stored in a module that
is called {\tt artisynth\_2\_0} for historical reasons. However, this
name does not reflect the actual version number, which is stored
in the file {\tt VERSION}. Once you have checked out {\tt artisynth\_2\_0},
you can rename the top level directory to anything you like.
\end{sideblock}

\subsection{Downloading Libraries}
\label{LibUpdateSec}

Because the {\tt .jar} files and native libraries used by ArtiSynth
are rather large, they are downloaded separately from the ArtiSynth
webserver. Typically, this is done automatically by the ArtiSynth
application, without requiring any action by the user.  If you run
ArtiSynth with the {\tt -updateLibs} command line option, the program
will ensure that not only are all the required libraries present, but
that they also match the latest versions on the server.

In cases where you obtain ArtiSynth by checking out a version from the
source repository, or when an update adds a new {\tt .jar} file, then
it may be necessary to run a stand-alone command to initially download
the libraries, because it won't be possible to compile the source
without the {\tt .jar} files.

On Linux, MacOS, and Windows systems running Cygwin, the command to
download libraries is called {\tt updateArtisynthLibs}, and is
located in {\tt \$ARTISYNTH\_HOME/bin}. You can run it by opening a
terminal window, changing to the {\tt ARTISYNTH\_HOME} directory, and
running the command

\begin{verbatim}
 > bin/updateArtisynthLibs
\end{verbatim}

If your {\tt PATH} environment variable has been set to include {\tt
\$ARTISYNTH\_HOME/bin} (see Section \ref{environmentVariablesSec}), you
can simply run the command

\begin{verbatim}
 > updateArtisynthLibs
\end{verbatim}

from any directory.

On Windows systems, the command
is called {\tt updateArtisynthLibs.bat}, and is located
in {\tt \$ARTISYNTH\_HOME\BKS bin}. You can execute the command
by double-clicking on it in a file-browser.
Alternatively, you can open a command interpreter ({\tt CMD}), navigate
to {\tt \$ARTISYNTH\_HOME}, and enter the command

\begin{verbatim}
 % bin\updateArtisynthLibs
\end{verbatim}

ArtiSynth libraries are stored under {\tt \$ARTISYNTH\_HOME/lib}, with
the {\tt .jar} files being placed in the {\tt lib} directory and the
native libraries in an appropriate subdirectory (e.g., {\tt Linux64} or
{\tt Windows}) which is created if necessary. The required libraries
are listed in the file {\tt \$ARTISYNTH\_HOME/lib/LIBRARIES}. This
file is checked into the repository, so the system can always
determine what libraries are needed for a particular checkout version.

\subsection{Environment variables}
\label{environmentVariablesSec}

The following environment variables are involved in ArtiSynth's execution:

\begin{description}

\item[ARTISYNTH\_HOME]\mbox{}
 
The path name of the ArtiSynth installation directory.

\item[ARTISYNTH\_PATH]\mbox{}

A list of directories, separated by colons ":" (or
semi-colons ";" on Windows), which ArtiSynth uses to search for
configuration files such as {\tt .artisyntInit} or {\tt .demoModels}.  A
typical setting for {\tt ARTISYNTH\_PATH} consists of the current directory
(indicated by "{\tt .}"), the user's home directory, and the ArtiSynth
installation directory. If {\tt ARTISYNTH\_PATH} is not defined explicitly
in the user's environment, ArtiSynth assumes an implicit path
consisting of the directory sequence just described.

\item[CLASSPATH]\mbox{}

A list of directories and/or {\tt .jar} files, separated by 
colons ":" (or semi-colons ";" on Windows) which Java uses to
locate its class files. This variable should be set to
include {\tt \$ARTISYNTH\_HOME/classes} and {\tt \$ARTISYNTH\_HOME/lib/*}
(the latter uses the wildcard {\tt *} to specify all the {\tt .jar} files
in {\tt \$ARTISYNTH\_HOME/lib}).

\item[PATH]\mbox{}
 
A list of directories, separated by colons ":" (or semi-colons ";" on
Windows), which the operating system uses to locate executable programs and
applications. Should be set to include {\tt \$ARTISYNTH\_HOME/bin}.  On Windows
and Cygwin, {\tt PATH} is also used to search for shared libraries,
and so should also be set to include
{\tt \$ARTISYNTH\_HOME/lib/Windows} (on 32 bit systems)
or {\tt \$ARTISYNTH\_HOME/lib/Windows64} (on 64 bit systems).

\item[LD\_LIBRARY\_PATH]\mbox{}

On Linux, a list of directories, separated by colons
":", which the operating system searches in order to find shared libraries.
Should be set to include {\tt \$ARTISYNTH\_HOME/lib/Linux} 
(for 32 bit systems) or {\tt \$ARTISYNTH\_HOME/lib/Linux64} (for 64 bit systems).

\item[DYLD\_LIBRARY\_PATH]\mbox{}

On MacOS, the equivalent of {\tt LD\_LIBRARY\_PATH}.
Should be set to include {\tt \$ARTISYNTH\_HOME/lib/Darwin-x86\_64}.

\item[OMP\_NUM\_THREADS]\mbox{}
 
Specifies the maximum number of processor cores that are available for
multicore execution. Setting this variable to the maximum number of
cores on your machine can significantly increase performance.

\end{description}

%JYTHON\_HOME::
%If Jython is installed on the system, should be set to the name of the
%Jython installation directory.

Note that settings for most of the above can be derived from the value
of {\tt ARTISYNTH\_HOME}.

The basic ArtiSynth execution commands described in
Section \ref{RunningArtisynthSec} will automatically set these environment
variables if they are not already set in the user's
environment. Nevertheless, we encourage all developers to set these
environment variables globally in order to provide maximum access to
all the ArtiSynth development tools and scripts. Different procedures
for doing this are described in the following sections.

\begin{sideblock}
{\bf Note:}\\
If you are using the Eclipse IDE, and you have not set the
above environment variables globally, then you will need to
set some of them directly in your Eclipse launch environment,
as described in Section \ref{EclipseEnvironmentVariablesSec}.
\end{sideblock}

%\label{sysdirSettingsSec}
%[cols="8,^1,^1",options="header"]
%|=============================================
%|Operating System | 32 bit | 64 bit |
%|Windows          | +Windows+ | N/A |
%|Linux            | +Linux+ | Linux64 |
%|MacOS            |  N/A  | Darwin-x86_64 |
%|=============================================

\subsection{Example environment setup for bash}
\label{bashEnvSec}

On Linux and MacOS, if you are using {\tt bash} as your shell, then the
environment can be configured by placing a block of commands similar
to the following in your {\tt .bashrc} file, located in your home
directory:

\begin{lstlisting}
# set ARTISYNTH_HOME to the appropriate location ...
export ARTISYNTH_HOME=$HOME/artisynth_2_X
export ARTISYNTH_PATH=.:$HOME:$ARTISYNTH_HOME

# For 32 bit Linux:
export LD_LIBRARY_PATH=$ARTISYNTH_HOME/lib/Linux:$LD_LIBRARY_PATH
# For 64 bit Linux:
#export LD_LIBRARY_PATH=$ARTISYNTH_HOME/lib/Linux64:$LD_LIBRARY_PATH
# For MacOS:
#export DYLD_LIBRARY_PATH=$ARTISYNTH_HOME/lib/Darwin-x86_64:$DYLD_LIBRARY_PATH

export CLASSPATH=$ARTISYNTH_HOME/classes:$ARTISYNTH_HOME/lib/'*':$CLASSPATH
export PATH=$ARTISYNTH_HOME/bin:$PATH
# Set to the number of cores on your machine:
export OMP_NUM_THREADS=2 
\end{lstlisting}

Be sure to set {\tt ARTISYNTH\_HOME} to the proper location of your
ArtiSynth installation directory, and comment or uncomment the
appropriate {\tt LD\_LIBRARY\_PATH} or {\tt DYLD\_LIBRARY\_PATH} section for your
machine.

These environment variables will be passed on to any program which you
run from the shell (such as {\tt artisynth} or {\tt eclipse}).  However, on
MacOS, they will not be passed on to programs (such as {\tt eclipse})
which you launch from the dock.

Alternatively, you can source the script {\tt setup.bash}, located in
the installation directory:

\begin{verbatim}
 > source setup.bash
\end{verbatim}

This will determine the system type automatically and set the
environment variables accordingly, with {\tt ARTISYNTH\_HOME} set to the
current directory from which the script is called (however,
it {\it won't} set {\tt OMP\_NUM\_THREADS}).

\subsection{Example environment setup for csh/tcsh}
\label{tcshEnvSec}

On Linux and MacOS, if you are using {\tt csh} or {\tt tcsh} as your
shell, then the environment can be configured by placing a block of
commands similar to the following
in your {\tt .cshrc} file, located in your home directory:

\begin{lstlisting}
# set ARTISYNTH_HOME to the appropriate location ...
setenv ARTISYNTH_HOME $HOME/artisynth_2_X
setenv ARTISYNTH_PATH .":"$HOME":"$ARTISYNTH_HOME

# For 32 bit Linux:
setenv LD_LIBRARY_PATH $ARTISYNTH_HOME/lib/Linux":"$LD_LIBRARY_PATH
# For 64 bit Linux:
#setenv LD_LIBRARY_PATH $ARTISYNTH_HOME/lib/Linux64":"$LD_LIBRARY_PATH
# For MacOS:
#setenv DYLD_LIBRARY_PATH $ARTISYNTH_HOME/lib/Darwin-x86_64":"$DYLD_LIBRARY_PATH

setenv CLASSPATH "$ARTISYNTH_HOME/classes:$ARTISYNTH_HOME/lib/*:$CLASSPATH"
setenv PATH $ARTISYNTH_HOME/bin":"$PATH
# Set to the number of cores on your machine:
setenv OMP_NUM_THREADS 2 
\end{lstlisting}

These environment variables will be passed on to any program which you
run from the shell (such as {\tt artisynth} or {\tt eclipse}). 

Alternatively, you can source the script {\tt setup.csh}, located in
the installation directory:

\begin{verbatim}
 > source setup.csh
\end{verbatim}

This will determine the system type automatically and set the
environment variables accordingly, with {\tt ARTISYNTH\_HOME} set to the
current directory from which the script is called (however,
it {\it won't} set {\tt OMP\_NUM\_THREADS}).

\subsection{Example environment setup on Windows}
\label{WindowsEnvironmentSec}

On Windows, it is recommended to set the environment variables in your
global user settings, as described in Section \ref{settingWindowsVariablesSec}. A
full set of variable settings would look something like this:

\begin{lstlisting}
ARTISYNTH_HOME c:\users\joe\artisynth_2_X
ARTISYNTH_PATH .;c:\users\joe;%ARTISYNTH_HOME%
CLASSPATH %ARTISYNTH_HOME%\classes;%ARTISYNTH_HOME%\lib\*
PATH %ARTISYNTH_HOME%\bin;%ARTISYNTH_HOME%\lib\Windows;%PATH%
OMP_NUM_THREADS 2
\end{lstlisting}

Of course, instead of {\tt \BKS users\BKS joe} and {\tt \BKS users\BKS joe\BKS artisynth\_2\_X} you
should use your home directory and the ArtiSynth install directory,
and {\tt OMP\_NUM\_THREADS} should be set to the maximum number of cores
available on your machine.

\subsection{Example environment setup on Cygwin}
\label{CygwinEnvironmentSec}

If you are using Windows Cygwin, then you can also set the environment
variables in {\tt .bashrc} for use in the bash shell.  This is done
similarly to Section \ref{bashEnvSec}, but with some differences: the shared
library directory is placed in {\tt PATH}, and {\tt ARTISYNTH\_HOME},
{\tt ARTISYNTH\_PATH}, and {\tt CLASSPATH} must be set to use the Windows path
style:

\begin{lstlisting}
# set AH to the location of the ArtiSynth install directory
AH=$HOME/artisynth_2_X

# For 32 bit Windows:
export PATH=$AH/bin:$AH/lib/Windows:$PATH
# For 64 bit Windows:
#export PATH=$AH/bin:$AH/lib/Windows64:$PATH

# Use Windows path style:
export ARTISYNTH_HOME=`cygpath -w $AH`
export ARTISYNTH_PATH=".;`cygpath -w $HOME`;$ARTISYNTH_HOME"
export CLASSPATH="$ARTISYNTH_HOME\classes;$ARTISYNTH_HOME\lib"'\*;'"$CLASSPATH"
# Set to the number of cores on your machine:
export OMP_NUM_THREADS=2
\end{lstlisting}

The scripts {\tt setup.bash} and {\tt setup.csh}, described above, are
configured to work properly for Cygwin.

\begin{sideblock}
{\bf Note:}\\
For Windows, it is still preferable to set environment variables
globally, as described in Section \ref{WindowsEnvironmentSec}, because setting them
in the Cygwin bash shell will not allow them to be seen when programs
(such as Eclipse) are launched directly from the desktop.
\end{sideblock}

%Special Notes for MacOS
%~~~~~~~~~~~~~~~~~~~~~~~
%
%You require Mac OSX 10.4 tiger and need to install j2sdk 1.5 and make
%it the default Java version:
%
%-------------------------------
%$ cd /System/Library/Frameworks/JavaVM.framework/Versions
%$ sudo mv CurrentJDK CurrentJDK.old
%$ sudo ln -s 1.5 CurrentJDK
%-------------------------------

\section{Using the Eclipse IDE}
\label{EclipseSec}

Many ArtiSynth users employ the Eclipse IDE for both code development
and running ArtiSynth.  This section describes how to import ArtiSynth
into Eclipse, and set the relevant environment variables.  It is
assumed that the user is familiar with both Eclipse and Java
programming; an introduction to these topics is beyond the scope of
this document.

Eclipse can be obtained from
\href{http://www.eclipse.org/downloads}{www.eclipse.org/downloads}.
The version that you want is (at the time of this writing) {\sf
Eclipse IDE for Java Developers}. Also, make sure that Eclipse is
configured to run Java 6 or newer; for details, see Section
\ref{checkingEclipseJavaVersionSec}.

If you are using a CVS check out of the development version (Section
\ref{obtainingDevelopmentVersionSec}), then you should first read
Section \ref{workingWithCVSSec}.

If you already have a version of ArtiSynth installed (either from a
distribution or CVS checkout), you can import this into Eclipse as a
project; see Section \ref{importingArtisynthSec}). Otherwise, you can
obtain a CVS checkout directly from within Eclipse; see Section
\ref{importingFromCVSSec}.

\begin{sideblock}

For historical reasons, the default Eclipse settings that are bundled
with ArtiSynth use a project name of {\tt artisynth\_2\_0}, even though this
does not reflect the current distribution number. If you wish, it is
possible to rename your project (to the current distribution number or
something else) after you have completed the installation procedures
described in either Section \ref{importingArtisynthSec} or
Section \ref{importingFromCVSSec}. Simply select the project in the {\sf Package Explorer}
window, and then choose {\sf Refactor > Rename} from the context menu.
\end{sideblock}

\subsection{Using Eclipse and CVS}
\label{workingWithCVSSec}

For general information on using CVS within eclipse, see
\href{http://wiki.eclipse.org/CVS\_FAQ}{http://wiki.eclipse.org/CVS\_FAQ}.

It is recommended that you ensure Eclipse works properly with external
CVS commands (this will be particularly necessary if you used the
{\sf cvs} command directly to obtain the checkout). To do this:

\begin{enumerate}

\item From within Eclipse, choose {\sf Window > Preferences} (or {\sf Eclipse > Preferences})

\item Navigate to {\sf Team > CVS > Ext Connection Method}.

\item Select {\sf Use another connection method type to connect}, and make
  sure this is set to {\tt extssh}.

\item Click {\sf OK}.

\end{enumerate}

\begin{sideblock}
{\tt extssh} is a connection protocol that is built in to eclipse, but
is not recognized by external CVS commands; it essentially means "use the {\tt ext}
protocal in conjunction with {\tt ssh} (secure shell)".
\end{sideblock}

\subsection{Importing ArtiSynth into Eclipse}
\label{importingArtisynthSec}

\begin{enumerate}

\item From {\bf outside} Eclipse, install the ArtiSynth Eclipse settings by
  unzipping {\tt \$ARTISYNTH\_HOME/support/eclipse/eclipse\-Settings.zip} into
  {\tt \$ARTISYNTH\_HOME}. This will create the files {\tt .project},
  {\tt .classpath}, and {\tt ArtiSynth.launch}, along with the directory
  {\tt .settings}, in {\tt \$ARTISYNTH\_HOME}.
  
  \begin{sideblock}
  {\bf Attention MacOS users:}\\[0.5em]
  The default zip utility on MacOS will create a new sub-folder called 
  {\tt eclipseSettings} and will extract the files there.  \emph{You do not want this!!}
  Some of the files are then labelled as ``hidden'' by MacOS, which will
  prevent you from moving them to the correct place manually. 
  Either extract the file directly to the {\tt\$ARTISYNTH\_HOME} directory 
  with a more standard application like {\sf 7-Zip} ({\sf 7zX} for OSX), 
  or use the {\tt unzip} utility from the command-line.
  \end{sideblock}


\item From within Eclipse, choose {\sf File > Import ...}.

\item In the {\sf Import} window, select {\sf General > Existing Projects into
  Workspace} and click {\sf Next}.

\item In the field {\sf Select root directory}, enter (or browse to) 
  {\tt \$ARTISYNTH\_HOME} and then click {\sf Finish}. 

\end{enumerate}

If Eclipse complains that {\sf "No projects are found to import"}, that most
likely means that {\tt eclipseSettings.zip} was not
properly unzipped into {\tt \$ARTISYNTH\_HOME}.

\subsection{Importing ArtiSynth directly from CVS}
\label{importingFromCVSSec}

It is also possible to import the ArtiSynth development version
directly from the CVS repository:

\begin{enumerate}

\item Choose {\sf File > Import} from the main menu.

\item Select {\sf CVS > Projects from CVS} and click {\sf Next}.

\item Click on {\sf Create a new repository location} and click {\sf Next}.

\item Fill in the repository information:

\begin{itemize}

\item Host: ssh.ece.ubc.ca

\item Repository Path: /ubc/ece/home/hct/other/hct/cvsroot

\item User: {\it your UBC ECE account name}

\item Password: {\it your UBC ECE account password}

\item Connection Type: Select {\tt extssh} if you don't intend to
access the project using external CVS commands. If you do wish to use
external CVS commands, select {\tt ext} instead; however, this may not
always work and may result in errors at the end of step 6.

\end{itemize}

\item Click {\sf Next}.

\item In the box next to {\sf Use specified module name}, type {\tt artisynth\_2\_0}
  and click {\sf  Next}

\item Select {\sf Check out as a project in the workspace} (with the project name
  {\tt artisynth\_2\_0}) and click {\sf Next}

\item Specify the location for the check out; this will become the
  ArtiSynth install directory {\tt \$ARTISYNTH\_HOME}. If you leave {\sf Use default
  workspace location} selected, this will be {\tt workspace/artisynth\_2\_0}, where
  {\tt workspace} is the Eclipse workspace directory. Otherwise, you
  can specify an explicit checkout location (which does not have to
  be located in the Eclipse workspace).

\item Click {\sf Finish}.

\item If necessary, open a Java perspective by choosing {\sf Window > Open
  Perspective > Java}. The project {\tt artisynth\_2\_0} should appear in
  the {\sf Package Explorer} window.

\item From {\bf outside} Eclipse, install the ArtiSynth Eclipse
settings by unzipping {\tt
\$ARTISYNTH\_HOME/support/eclipse/eclipse\-Settings.zip} into {\tt
\$ARTISYNTH\_HOME}. (This will create the files {\tt .classpath} and
{\tt ArtiSynth.launch}, along with the directory {\tt .settings}, in
{\tt \$ARTISYNTH\_HOME}, and will overwrite the existing {\tt
.project} file).  

\begin{sideblock}
  {\bf Attention MacOS users:}\\[0.5em]
  This is where most installs fail.
  
  The default zip utility on MacOS will create a new sub-folder called 
  {\tt eclipseSettings} and will extract the files there.  \emph{You do not want this!!}
  Some of the files are then labelled as ``hidden'' by MacOS, which will
  prevent you from moving them to the correct place manually. 
  Either extract the file directly to the {\tt\$ARTISYNTH\_HOME} directory 
  with a more standard application like {\sf 7-Zip} ({\sf 7zX} for OSX), 
  or use the {\tt unzip} utility from the command-line.  For the latter,
  open a terminal window, change to the ArtiSynth install directory,
  enter the command
  \begin{verbatim}
    unzip support/eclipse/eclipseSettings.zip
  \end{verbatim}
  and respond with {\tt y} when asked if it is OK to replace {\tt .project}.
\end{sideblock}

\item From {\bf outside} Eclipse, download the required
{\tt .jar} files and native libraries using the 
{\tt updateArtisynthLibs} command as described in Section
\ref{LibUpdateSec}.

\item Finally, load the new settings into the project by selecting the project
{\tt artisynth\_2\_0} in the {\sf Package Explore} window and selecting
{\sf Refresh} from the context menu.

\end{enumerate}

\subsection{Configuring environment variables}
\label{EclipseEnvironmentVariablesSec}

It may be necessary to set certain environment variables directly in
your Eclipse launch configuration (as described below) so that
ArtiSynth can locate configuration files and native library support.

If you are running on Windows, and you have set the environment
variables as described in Section \ref{WindowsEnvironmentSec}, then
you can skip this section. If you are running on Linux or MacOS, and
you are starting eclipse directly from a terminal window, you can also
skip this section.  Otherwise, you will need to set certain
environment variables directly in your Eclipse launch configuration
(as described below) so that ArtiSynth can locate configuration files
and native library support.  

\begin{sideblock}
{\bf Note:} At present, eclipse does not expand environment variables.
In all the variable settings described below, references to {\tt
\$ARTISYNTH\_HOME} should be expanded (manually) to the path of the
ArtiSynth install directory.
\end{sideblock}

\begin{itemize}

\item On all systems, you will need to set {\tt ARTISYNTH\_HOME} to the path of
the ArtiSynth installation directory.  You may also optionally set
{\tt ARTISYNTH\_PATH} as described in Section \ref{environmentVariablesSec}, and you
should set {\tt OMP\_NUM\_THREADS} to the maximum number of cores on your
machine if you want the system to take advantage of multiple cores.

%If Jython is installed on your system, you should set {\tt JYTHON\_HOME} to
%the path of the Jython install directory.



\item On Windows, you will need to set {\tt PATH} to
{\tt \$ARTISYNTH\_HOME\BKS lib\BKS Windows} (32 bit systems) or \\
{\tt \$ARTISYNTH\_HOME\BKS lib\BKS Windows64} (64 bit systems).

\item On Linux, you will need to set {\tt LD\_LIBRARY\_PATH} to
{\tt \$ARTISYNTH\_HOME/lib/Linux} (32 bit systems) or \\
{\tt \$ARTISYNTH\_HOME/lib/Linux64} (64 bit systems).

\item On MacOS, you will need to set {\tt DYLD\_LIBRARY\_PATH} to
{\tt \$ARTISYNTH\_HOME/lib/Darwin-x86\_64}.

\end{itemize}

\begin{sideblock}
{\bf Note:} There is a bug in Eclipse 4.0+ on Windows where it will replace 
the system's native {\tt \%PATH\%} variable rather than appending to it.  
To correct this behaviour, define {\tt PATH} as 
{\tt \$\{env\_var:path\};\$ARTISYNTH\_HOME\BKS lib\BKS Windows64} 
\end{sideblock}

To set environment variables within Eclipse:

\begin{enumerate}

\item Open a java perspective if necessary by choosing
  {\sf Window > Open Perspective > Java}.

\item Select the artisynth project in the {\sf Package Explorer} form.

\item Choose {\sf Run > Run Configurations...} to open the {\sf Run
  Configurations} window.

\item In the left panel, under {\sf Java Application}, select {\sf ArtiSynth}.

\item In the right panel, select the {\sf environment} tab.

\item To create a new environment variable, click the {\sf New} button and
  enter the name and value in the dialog box.

\item When finished, make sure that {\sf Append environment to native
  environment} is selected, and click {\sf Apply}.

\end{enumerate}

\subsection{Building and Running ArtiSynth}
\label{BuildingAndRunningArtisynthSec}

Once ArtiSynth has been installed, you should be able to build and run
the system. If necessary, first open a Java perspective by choosing
{\sf Window > Open Perspective > Java}. The project {\tt artisynth\_2\_0} (or
whatever you have renamed it to) should appear in the {\sf Package
Explorer} window.

To build the system, select the project {\tt artisynth\_2\_0} in the
{\sf Package Explorer} window, and then choose {\sf Project > Build Project}
(it may be necessary to deselect {\sf Build Automatically} in order to
enable {\sf Build Project}).  To run the system, choose {\sf Run > Run}.

If you are using a CVS checkout of the development version,
then you can obtain the latest updates from the repository
by selecting the project in the {\sf Package Explorer} and selecting
{\sf Team > Update} from the context menu.

\subsection{Checking that Eclipse uses Java 6 or newer }
\label{checkingEclipseJavaVersionSec} 

If you are unsure that Eclipse is using Java 6 or newer, then:

\begin{enumerate}

\item Choose {\sf Window > Preferences} (or {\sf Eclipse > Preferences}).

\item Select {\sf Java > Installed JREs}.

\item Check the installed JREs in the table, under the {\sf Name} field.  Make
sure that Java 6 is present (this will likely be indicated by a name
containing {\tt jdk-1.6.0.xx}, where {\tt xx} is an update number, or, on
Windows, {\tt jre6}).

\item If Java 6 is not present, first make sure that it is installed on
your system (see Section \ref{externalSoftwareSec}), and then add it to the list of
installed JREs by clicking {\sf Search} in the Preferences window,
navigating to the Java 6 install directory, and clicking {\sf OK}.

\item Make sure that Java 6 is selected from the list of installed JREs
(via the checkbox on the left).

\end{enumerate}

\subsection{Preventing excessive resource copying}

By default, ArtiSynth classes are built in a directory tree
({\tt \$ARTISYNTH\_HOME/classes}) that is separate from the source tree
({\tt \$ARTISYNTH\_HOME/src}). That means that Eclipse will try to copy all
non-Java files and directories from the source tree into the build
tree. For ArtiSynth, this is excessive, and results in many files
being copied that don't need to be, since ArtiSynth looks for
resources in the source tree anyway.

On MacOS and Linux, you can nhibit most of this copying:

\begin{enumerate}

\item Choose {\sf Window > Preferences} (or {\sf Eclipse > Preferences}).

\item Select {\sf Java > Compiler > Building}.

\item Open {\sf Output folder}, and in the box entitled {\sf Filter resources},
  enter the string:

\begin{lstlisting}
    Makefile,*.l*,*.?,*.??,*.???,*.????,???,????,?????
\end{lstlisting}

\end{enumerate}

That should filter out the copying of most non-java files.

On Windows, Eclipse doesn't support {\tt ?} as a filter character, so
you need to enter something like this instead:

\begin{lstlisting}
    Makefile,*.a*,*.b*,*.c,*.cc,*.cxx,*.e*,*.g*,*.h*,*.i*,*.m*,*.n*,*.o*,*.r*,*.s*,*.t*
\end{lstlisting}

Or, to prevent copying any resource, simply enter: 
\begin{lstlisting}
    *
\end{lstlisting}

%\subsection{Importing the Eclipse code style file}
%
%There is a code style file for developers using Eclipse located at  
%{\tt \$ARTISYNTH\_HOME/support/eclipse/artisynthCodeFormat.xml}. Please be sure to read
%
%the %Coding Standard% file on the ArtiSynth website as some of the formatting cannot
%be done automatically by Eclipse.
%
%. Select the artisynth project under the %Package Explorer% form.
%. Navigate to %Project->Properties->Java Code Style->Formatter%.
%. Check the box labeled %Enable Project Specific Settings%.
%. In the %Formatter% panel, click the %Import% button. This will open a
%%Import Profiles% panel.
%. Navigate using the panel to the {\tt \$ARTISYNTH\_HOME/support/eclipse/} directory.\
%. Select the {\tt artisynthCodeFormat.xml} file.
%. Click %OK% button in the %Import Profiles% panel.
%. Click %OK% button in the %Properties% panel.

%\section{JOGL Version Clashes}
%\label{JOGLClashesSec}
%
%ArtiSynth uses the version of JOGL described
%in the file {\tt \$ARTISYNTH\_HOME/lib/JOGL\_VERSION}. If there are two versions
%present, then one version may reference the other's files by
%mistake. To prevent this, it is typically best to remove the local
%version and use the version included with ArtiSynth. If this is not
%possible, then it may be possible to suppress one version by carefully
%controlling the path the computer searches. For Windows, the search
%path is declared by the the {\tt PATH} variable. For Linux and Mac OS, the 
%path is named {\tt LD\_LIBRARY\_PATH} and {\tt DYLD\_LIBRARY\_PATH} respectively.
%
%  - To suppress the local version with the version included with
%    ArtiSynth, place {\tt \$ARTISYNTH\_HOME/lib/Linux} at the front of the
%    search path.
%
%  - To suppress the ArtiSynth version with the local version, replace
%    the {\tt \$ARTISYNTH\_HOME/lib/jogl.jar} within the {\tt CLASSPATH} variable
%    with the local version's path, and place the directory of the
%    corresponding libraries at the front of the search path. This
%    directory is typically {\tt /usr/local/lib}.
%
%The JOGL dynamic libraries that become confused are {\tt jogl} and {\tt jogl\_awt}. The
%file extensions depend upon the operating system. For Windows, the file 
%extensions are {\tt .dll}. For Linux and Mac, the file extensions are {\tt .so}
%and {\tt .jnilib} respectively.
%
%One case where a JOGL conflict can occur is if Matlab is also installed.
%Executing the {\tt matlabroot} command within Matlab provides Matlab's root
%directory. Within a subdirectory of this file's bin file is the JOGL files.
%This error occurred while opening ArtiSynth. Several partially painted windows 
%opened, then the program hanged and output the following stack trace. 
%
%\begin{lstlisting}
%Exception in thread "AWT-EventQueue-0" java.lang.RuntimeException:
%  java.lang.IllegalArgumentException: argument type mismatch
%   at com.sun.opengl.impl.JAWT_DrawingSurfaceInfo.newPlatformInfo...
%   at com.sun.opengl.impl.JAWT_DrawingSurfaceInfo.platformInfo...
%   at com.sun.opengl.impl.windows.WindowsOnscreenGLDrawable.lockSurface...
%   at com.sun.opengl.impl.windows.WindowsOnscreenGLContext.makeCurrentImpl...
%   at com.sun.opengl.impl.GLContextImpl.makeCurrent...
%   ... 16 more
%Caused by: java.lang.IllegalArgumentException: argument type mismatch
%   at sun.reflect.NativeMethodAccessorImpl.invoke0...
%   at sun.reflect.NativeMethodAccessorImpl.invoke...
%   at sun.reflect.DelegatingMethodAccessorImpl.invoke...
%   at java.lang.reflect.Method.invoke...
%   at com.sun.opengl.impl.JAWT_DrawingSurfaceInfo.newPlatformInfo...
%   ... 20 more
%\end{lstlisting}

%\section{Installing Jython}
%\label{JythonInstallationSec}
%
%Jython can be installed by going to http://www.jython.org and
%following the download and installation instructions. On operating
%systems with {\tt apt-get}, it is easier to install Jython with the
%command:
%
%\begin{verbatim}
%  % sudo apt-get install jython
%\end{verbatim}
%
%After, create an environmental variable named {\tt JYTHON_HOME}, and set
%it to the root of the install directory. ArtiSynth checks if this
%variable exists to determine whether or not Jython is installed.

%\label{userAndHostNamesSec}
%Determining your user and host names
%~~~~~~~~~~~~~~~~~~~~~~~~~~~~~~~~~~~~
%
%Mac:: 
%Open a console window use the commands {\tt id -p} and {\tt hostname} to
%determine the user and host names, respectively
%
%Linux:: 
%Open a console window use the commands {\tt whoami} and {\tt hostname} to
%determine the user and host names, respectively
%
%Windows:: 
%Your user name is the name of your directory in {\tt C:\Documents and
%Settings\} or {\tt C:\Users\}, and the hostname is the can be found by 
%typing {\tt hostname} in a command window.

\section{Installing Cygwin}
\label{CygwinInstallationSec}

Cygwin provides Windows users with a Linux-like, shell-based command
environment.  Its provides useful tools and enables ArtiSynth users to
employ all the script-based commands in {\tt \$ARTISYNTH\_HOME/bin}, as well
as various {\tt Makefile} commands.

Cygwin can be downloaded from 
\href{http://www.cygwin.com}{www.cygwin.com}. Run the
executable that was just downloaded (setup.exe) to begin
installation. After selecting a download source, an install directory,
a package directory, an internet connection, and a download site, the
installer will display a list of packages which the user can select
for download. Normally, a default set of packages is already selected
for installation. It is advisable to install these packages to ensure
that Cygwin retains its basic capabilities. It is also recommended
that you select the following additional packages:

\begin{itemize}

\item {\tt archive}
\item {\tt asciidoc}, {\tt cvs}, {\tt make} (located under {\tt devel})
\item {\tt python} (located under {\tt interpreters}).
\item {\tt openssh} (located under {\tt net})

\end{itemize}

If you are planning to compile C/C++ code, you may also want to
install the various {\tt gcc} and {\tt gdb} packages (located under {\tt devel}).

\section{Setting Windows Environment Variables}
\label{settingWindowsVariablesSec}

A user can view, set, or change environment variables on Windows via
the following steps:

\begin{enumerate}

\item Right-click {\sf My Computer}, and then click {\sf Properties}.
\item Click the {\sf Advanced} tab.
\item Click {\sf Environment variables}.
\item Choose one of the following options:

\begin{itemize}

\item Click {\sf New} to add a new variable name and value.
\item Click an existing variable, and then {\sf Edit} to change its name or value.
\item Click an existing variable, and then {\sf Delete} to remove it.

\end{itemize}
\end{enumerate}

Variable settings can reference other environment variables, by
surrounding them with percent signs, as in {\tt \%VARIABLE\_NAME\%}.  For
example, suppose you already have an environment variable {\tt HOME} that
gives the location of your home directory, and your ArtiSynth
distribution is located in {\tt packages\BKS artisynth\_2\_X} relative to your
home directory. Then the environment variable {\tt ARTISYNTH\_HOME} can be
specified as

\begin{verbatim}
  %HOME%\packages\artisynth_2_X
\end{verbatim}

\subsection{The Windows PATH variable}
\label{settingWindowsPathSec}

The special environment variable {\tt PATH} tells the system where to find
applications and programs. It consists of a list of directory names
separated by semicolons ";". If you have applications or programs that
reside in non-standard locations, you can enable the system to find
them by adding their containing directories to the existing {\tt PATH}
value. For example, the programs associated with ArtiSynth reside in
{\tt \$ARTISYNTH\_HOME\BKS bin}. If {\tt ARTISYNTH\_HOME} has the value
{\tt "c:\BKS packages\BKS artisynth\_2\_X"}, then the ArtiSynth programs
can be made visible to the system by setting {\tt PATH} to

\begin{verbatim}
  %PATH%;c:\packages\artisynth_2_X\bin
\end{verbatim}

Alternatively, if the ArtiSynth home location is described by
the environment variable {\tt ARTISYNTH\_HOME}, then the
above can be expressed as

\begin{verbatim}
  %PATH%;%ARTISYNTH_HOME%\bin
\end{verbatim}

Note that most programs and applications need to be restarted in order
to get them to notice changes to the {\tt PATH}.

\end{document}
