\section{Supporting classes}

ArtiSynth uses a large number of supporting classes, mostly defined in
the super package {\tt maspack}, for handling mathematical and
geometric quantities. Those that are refered to in this manual are
summarized in this section.

\subsection{Vectors and matrices}

Among the most basic classes are those used to implement vectors and
matrices, defined in {\tt mapack.matrix}. All vector classes implement
the interface \javaclass[maspack.matrix]{Vector} and all matrix
classes implement \javaclass[maspack.matrix]{Matrix}, which provide a
number of standard methods for setting and accessing values and
reading and writing from I/O streams. 

General sized vectors and matrices are implemented by
\javaclass[maspack.matrix]{VectorNd} and
\javaclass[maspack.matrix]{MatrixNd}. These provide all the usual
methods for linear algebra operations such as addition, scaling, and
multiplication:
%
\begin{lstlisting}[]
  VectorNd v1 = new VectorNd (5);        // create a 5 element vector
  VectorNd v2 = new VectorNd (5); 
  VectorNd vr = new VectorNd (5); 
  MatrixNd M = new MatrixNd (5, 5);      // create a 5 x 5 matrix

  M.setIdentity();                       // M = I
  M.scale (4);                           // M = 4*M

  v1.set (new double[] {1, 2, 3, 4, 5}); // set values
  v2.set (new double[] {0, 1, 0, 2, 0});
  v1.add (v2);                           // v1 += v2
  M.mul (vr, v1);                        // vr = M*v1

  System.out.println ("result=" + vr.toString ("%8.3f"));
\end{lstlisting}
%
As illustrated in the above example, vectors and matrices both provide
a {\tt toString()} method that allows their elements to be formated
using a C-printf style format string. This is useful for providing
concise and uniformly formatted output, particularly for diagnostics.
The output from the above example is
%
\begin{verbatim}
  result=   4.000   12.000   12.000   24.000   20.000
\end{verbatim}
%
Detailed specifications for the format string are provided in the
documentation for \javamethod[maspack.util]{NumberFormat.set(String)}.
If either no format string, or the string {\tt "\%g"}, is specified,
{\tt toString()} formats all numbers using the full-precision output
provided by {\tt Double.toString(value)}.

For computational efficiency, a number of fixed-size vectors and
matrices are also provided. The most commonly used are those defined
for three dimensions, including \javaclass[maspack.matrix]{Vector3d}
and \javaclass[maspack.matrix]{Matrix3d}:
%
\begin{lstlisting}[]
  Vector3d v1 = new Vector3d (1, 2, 3);
  Vector3d v2 = new Vector3d (3, 4, 5);
  Vector3d vr = new Vector3d ();
  Matrix3d M = new Matrix3d();

  M.set (1, 2, 3,  4, 5, 6,  7, 8, 9);

  M.mul (vr, v1);        // vr = M * v1
  vr.scaledAdd (2, v2);  // vr += 2*v2;
  vr.normalize();        // normalize vr
  System.out.println ("result=" + vr.toString ("%8.3f"));
\end{lstlisting}
%

\subsection{Rotations and transformations}
\label{RigidTransform3d:sec}

{\tt maspack.matrix} contains a number classes that implement rotation
matrices, rigid transforms, and affine transforms. 

Rotations (Section \ref{Rotations:sec}) are commonly described using a
\javaclass[maspack.matrix]{RotationMatrix3d}, which implements a
rotation matrix and contains numerous methods for setting rotation
values and transforming other quantities. Some of the more commonly
used methods are:
%
\begin{lstlisting}[]
   RotationMatrix3d();         // create and set to the identity
   RotationMatrix3d(u, angle); // create and set using an axis-angle

   setAxisAngle (u, ang);      // set using an axis-angle
   setRpy (roll, pitch, yaw);  // set using roll-pitch-yaw angles
   setEuler (phi, theta, psi); // set using Euler angles
   invert ();                  // invert this rotation
   mul (R)                     // post multiply this rotation by R
   mul (R1, R2);               // set this rotation to R1*R2
   mul (vr, v1);               // vr = R*v1, where R is this rotation
\end{lstlisting}
%
Rotations can also be described by
\javaclass[maspack.matrix]{AxisAngle}, which characterizes a rotation
as a single rotation about a specific axis.

Rigid transforms (Section \ref{RigidTransforms:sec}) are used by
ArtiSynth to describe a rigid body's pose, as well as its relative
position and orientation with respect to other bodies and coordinate
frames.  They are implemented by
\javaclass[maspack.matrix]{RigidTransform3d}, which exposes its
rotational and translational components directly through the fields
{\tt R} (a {\tt RotationMatrix3d}) and {\tt p} (a {\tt
Vector3d}). Rotational and translational values can be set and
accessed directly through these fields.  In addition, {\tt
RigidTransform3d} provides numerous methods, some of the more commonly
used of which include:
%
\begin{lstlisting}[]
   RigidTransform3d();         // create and set to the identity
   RigidTransfrom3d(x, y, z);  // create and set translation to x, y, z

   // create and set translation to x, y, z and rotation to roll-pitch-yaw
   RigidTransfrom3d(x, y, z, roll, pitch, yaw);

   invert ();                  // invert this transform
   mul (T)                     // post multiply this transform by T
   mul (T1, T2);               // set this transform to T1*T2
   mulLeftInverse (T1, T2);    // set this transform to inv(T1)*T2
\end{lstlisting}
%

Affine transforms (Section \ref{AffineTransforms:sec}) are used by
ArtiSynth to effect scaling and shearing transformations on
components. They are implemented by
\javaclass[maspack.matrix]{AffineTransform3d}.

Rigid transformations are actually a specialized form of affine
transformation in which the basic transform matrix equals a rotation.
{\tt RigidTransform3d} and {\tt AffineTransform3d} hence both derive
from the same base class
\javaclass[maspack.matrix]{AffineTransform3dBase}.

\subsection{Points and Vectors}

The rotations and transforms described above can be used to transform
both vectors and points in space.

Vectors are most commonly implemented using
\javaclass[maspack.matrix]{Vector3d}, while points can be implemented
using the subclass \javaclass[maspack.matrix]{Point3d}.  The only
difference between {\tt Vector3d} and {\tt Point3d} is that the former
ignores the translational component of rigid and affine transforms;
i.e., as described in Sections \ref{RigidTransforms:sec} and
\ref{AffineTransforms:sec}, a vector {\tt v} has
an implied homogeneous representation of
%
\begin{equation}
\v^* \equiv \matl \v \\ 0 \matr,
\end{equation}
%
while the representation for a point {\tt p} is
%
\begin{equation}
\p^* \equiv \matl \p \\ 1 \matr.
\end{equation}
%

Both classes provide a number of methods for applying rotational and
affine transforms. Those used for rotations are
%
\begin{lstlisting}[]
  void transform (R);             // this = R * this
  void transform (R, v1);         // this = R * v1
  void inverseTransform (R);      // this = inverse(R) * this
  void inverseTransform (R, v1);  // this = inverse(R) * v1
\end{lstlisting}
%
where {\tt R} is a rotation matrix and {\tt v1} is a vector (or a point
in the case of {\tt Point3d}).

The methods for applying rigid or affine transforms include:
\begin{lstlisting}[]
  void transform (X);             // transforms this by X         
  void transform (X, v1);         // sets this to v1 transformed by X
  void inverseTransform (X);      // transforms this by the inverse of X
  void inverseTransform (X, v1);  // sets this to v1 transformed by inverse of X
\end{lstlisting}
where {\tt X} is a rigid or affine transform.
As described above, in the case of {\tt Vector3d}, these methods
ignore the translational part of the transform and apply only the
matrix component ({\tt R} for a {\tt RigidTransform3d} and {\tt A} for
an {\tt AffineTransform3d}).
In particular, that means that for a {\tt RigidTransform3d} given by {\tt X}
and a {\tt Vector3d} given by {\tt v},
the method calls
%
\begin{lstlisting}[]
  v.transform (X.R)
  v.transform (X)
\end{lstlisting}
%
produce the same result.

\subsection{Spatial vectors and inertias}
\label{SpatialVectors:sec}

The velocities, forces and inertias associated with 3D coordinate
frames and rigid bodies are represented using the 6 DOF spatial
quantities described in Sections \ref{SpatialVelocitiesAndForces:sec}
and \ref{SpatialInertia:sec}. These are implemented by classes in the
package {\tt maspack.spatialmotion}.

Spatial velocities (or twists) are implemented by
\javaclass[maspack.spatialmotion]{Twist}, which exposes its
translational and angular velocity components through the publicly
accessible fields {\tt v} and {\tt w}, while spatial forces (or
wrenches) are implemented by
\javaclass[maspack.spatialmotion]{Wrench}, which exposes its
translational force and moment components through the publicly
accessible fields {\tt f} and {\tt m}.

Both {\tt Twist} and {\tt Wrench} contain methods for algebraic
operations such as addition and scaling. They also contain {\tt
transform()} methods for applying rotational and rigid transforms.
The rotation methods simply transform each component by the supplied
rotation matrix. The rigid transform methods, on the other hand,
assume that the supplied argument represents a transform between two
frames fixed within a rigid body, and transform the twist or wrench
accordingly, using either (\ref{XvelAB:eqn}) or (\ref{XforceAB:eqn}).

The spatial inertia for a rigid body is implemented by
\javaclass[maspack.spatialmotion]{SpatialInertia}, which contains a
number of methods for setting its value given various mass, center of
mass, and inertia values, and querying the values of its components.
It also contains methods for scaling and adding, transforming between
coordinate systems, inversion, and multiplying by spatial vectors.

\subsection{Meshes}
\label{Meshes:sec}

ArtiSynth makes extensive use of 3D meshes, which are defined in {\tt
maspack.geometry}.  They are used for a variety of purposes, including
visualization, collision detection, and computing physical properties
(such as inertia or stiffness variation within a finite element
model).

A mesh is essentially a collection of vertices
(i.e., points) that are topologically connected in some way.  All
meshes extend the abstract base class
\javaclass[maspack.geometry]{MeshBase}, which supports the vertex
definitions, while subclasses provide the topology.

Through {\tt MeshBase}, all meshes provide methods for
adding and accessing vertices. Some of these include:
%
\begin{lstlisting}[]
  int getNumVertices();              // return the number of vertices
  Vertex3d getVertex (int idx);      // return the idx-th vertex
  void addVertex (Vertex3d vtx);     // add vertex vtx to the mesh
  Vertex3d addVertex (Point3d p);    // create and return a vertex at position p
  void removeVertex (Vertex3d vtx);  // remove vertex vtx for the mesh
  ArrayList<Vertex3d> getVertices(); // return the list of vertices
\end{lstlisting}
%
Vertices are implemented by \javaclass[maspack.geometry]{Vertex3d},
which defines the position of the vertex (returned by the method
\javamethod*[maspack.geometry.Vertex3d]{getPosition()}), and also
contains support for topological connections. In addition, each vertex
maintains an index, obtainable via
\javamethod*[maspack.geometry.Vertex3d]{getIndex()}, that equals the
index of its location within the mesh's vertex list. This makes it
easy to set up parallel array structures for augmenting mesh vertex
properties.

Mesh subclasses currently include:

\begin{description}

\item[\protect{\javaclass[maspack.geometry]{PolygonalMesh}}]\mbox{}

Implements a 2D surface
mesh containing faces implemented using half-edges.

\item[\protect{\javaclass[maspack.geometry]{PolylineMesh}}]\mbox{}

Implements a mesh
consisting of connected line-segments (polylines).

\item[\protect{\javaclass[maspack.geometry]{PointMesh}}]\mbox{}

Implements a point cloud with
no topological connectivity.

\end{description}

\javaclass[maspack.geometry]{PolygonalMesh} is used quite extensively
and provides a number of methods for implementing faces, including:
%
\begin{lstlisting}[]
  int getNumFaces();              // return the number of faces
  Face getFace (int idx);         // return the idx-th face
  Face addFace (int[] vidxs);     // create and add a face from specified vertex indices
  void removeFace (Face f);       // remove the face f
  ArrayList<Face> getFaces();     // return the list of faces
\end{lstlisting}
%
The class \javaclass[maspack.geometry]{Face} implements a face as a
counter-clockwise arrangement of vertices linked together by
half-edges (class \javaclass[maspack.geometry]{HalfEdge}).
{\tt Face} also supplies a face's (outward facing) normal
via 
\javamethod[maspack.geometry.Face]{getNormal()}.

Some mesh uses within ArtiSynth, such as collision detection, require a
{\it triangular} mesh; i.e., one where all faces have three vertices.
The method \javamethod[maspack.geometry.PolygonalMesh]{isTriangular()}
can be used to check for this. Meshes that are not triangular can be
made triangular using 
\javamethod[maspack.geometry.PolygonalMesh]{triangulate()}.

\subsubsection{Mesh creation}

It is possible to create a mesh by direct construction. For example,
the following code fragment creates a simple closed tetrahedral
surface:
%
\begin{lstlisting}[]
   // a simple four-faced tetrahedral mesh 
   PolygonalMesh mesh = new PolygonalMesh();
   mesh.addVertex (0, 0, 0);
   mesh.addVertex (1, 0, 0);
   mesh.addVertex (0, 1, 0);
   mesh.addVertex (0, 0, 1);
   mesh.addFace (new int[] { 0, 2, 1 });
   mesh.addFace (new int[] { 0, 3, 2 });
   mesh.addFace (new int[] { 0, 1, 3 });
   mesh.addFace (new int[] { 1, 2, 3 });      
\end{lstlisting}
%

However, meshes are more commonly created using either one of the
factory methods supplied by \javaclass[maspack.geometry]{MeshFactory},
or by reading a definition from a file (Section \ref{MeshFileIO:sec}).

Some of the more commonly used factory methods for creating polyhedral
meshes include:
%
\begin{lstlisting}[]
  MeshFactory.createSphere (radius, nslices, nlevels);
  MeshFactory.createBox (widthx, widthy, widthz);
  MeshFactory.createCylinder (radius, height, nslices);
  MeshFactory.createPrism (double[] xycoords, height);
  MeshFactory.createTorus (rmajor, rminor, nmajor, nminor);
\end{lstlisting}
%
Each factory method creates a mesh in some standard coordinate
frame. After creation, the mesh can be transformed using the
\javamethodAlt{maspack.geometry.MeshBase.transform(AffineTransform3dBase)}%
{transform(X)} method, where {\tt X} is either a rigid transform (
\javaclass[maspack.matrix]{RigidTransform3d}) or a more general affine
transform (\javaclass[maspack.matrix]{AffineTransform3d}).
For example, to create a rotated box centered on $(5, 6, 7)$,
one could do:
%
\begin{lstlisting}[]
  // create a box centered at the origin with widths 10, 20, 30:
  PolygonalMesh box = MeshFactor.createBox (10, 20, 20);

  // move the origin to 5, 6, 7 and rotate using roll-pitch-yaw
  // angles 0, 0, 45 degrees:
  box.transform (
     new RigidTransform3d (5, 6, 7,  0, 0, Math.toRadians(45)));
\end{lstlisting}
%
One can also scale a mesh using
\javamethodAlt{maspack.geometry.MeshBase.scale(double)}{scale(s)},
where {\tt s} is a single scale factor, or
\javamethodAlt{maspack.geometry.MeshBase.scale(double,double,double)}%
{scale(sx,sy,sz)}, where {\tt sx}, {\tt sy}, and {\tt sz} are separate
scale factors for the x, y and z axes. This provides a useful way to
create an ellipsoid:
%
\begin{lstlisting}[]
   // start with a unit sphere with 12 slices and 6 levels ...
  PolygonalMesh ellipsoid = MeshFactor.createSphere (1.0, 12, 6);

  // and then turn it into an ellipsoid by scaling about the axes:
  ellipsoid.scale (1.0, 2.0, 3.0);
\end{lstlisting}
%
\javaclass[maspack.geometry]{MeshFactory} can also be used to create
new meshes by performing boolean operations on existing ones:
%
\begin{lstlisting}[]
  MeshFactory.getIntersection (mesh1, mesh2);
  MeshFactory.getUnion (mesh1, mesh2);
  MeshFactory.getSubtraction (mesh1, mesh2);
\end{lstlisting}
%

\subsubsection{Reading and writing mesh files}
\label{MeshFileIO:sec}

The package {\tt maspack.geometry.io} supplies a number of classes for
writing and reading meshes to and from files of different formats.

Some of the supported formats and their associated readers and writers
include:

\begin{tabular}{|lll|}
\hline
Extension & Format & Reader/writer classes \\
\hline
.obj & Alias Wavefront & \tt WavefrontReader, WavefrontWriter \\
.ply & Polygon file format & \tt PlyReader, PlyWriter \\
.stl & STereoLithography & \tt StlReader, StlWriter \\
.gts & GNU triangulated surface & \tt GtsReader, GtsWriter \\
.off & Object file format & \tt OffReader, OffWriter \\
\hline
\end{tabular}

The general usage pattern for these classes is to construct the
desired reader or writer with a path to the desired file, and then
call {\tt readMesh()} or {\tt writeMesh()} as appropriate:
%
\begin{lstlisting}[]
   // read a mesh from a .obj file:
   WavefrontReader reader = new WavefrontReader ("meshes/torus.obj");
   PolygonalMesh mesh = null;
   try {
      mesh = reader.readMesh();
   }
   catch (IOException e) {
      System.err.println ("Can't read mesh:");
      e.printStackTrace();
   }
\end{lstlisting}
%
Both {\tt readMesh()} and {\tt writeMesh()} may throw I/O exceptions,
which must be either caught, as in the example above, or
thrown out of the calling routine.

For convenience, one can also use the classes
\javaclass[maspack.geometry.io]{GenericMeshReader} or
\javaclass[maspack.geometry.io]{GenericMeshWriter}, which internally
create an appropriate reader or writer based on the file
extension. This enables the writing of code
that does not depend on the file format:
%
\begin{lstlisting}[]
   String fileName;
   ...
   PolygonalMesh mesh = null;
   try {
      mesh = (PolygonalMesh)GenericMeshReader.readMesh(fileName);
   }
   catch (IOException e) {
      System.err.println ("Can't read mesh:");
      e.printStackTrace();
   }
\end{lstlisting}
%
Here, {\tt fileName} can refer to a mesh of any format supported by
{\tt GenericMeshReader}. Note that the mesh returned by {\tt
readMesh()} is explicitly cast to {\tt PolygonalMesh}.  This is
because {\tt readMesh()} returns the superclass {\tt MeshBase}, since
the default mesh created for some file formats may be different from
{\tt PolygonalMesh}.
